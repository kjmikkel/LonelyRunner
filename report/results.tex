\section{Results}
\label{results}
\todo{pick results}
\subsection{Introduction}
In this section I will show and discuss the most important results I have found while testing the Geometrical and Numerical algorithms. 

I will first show, and comment, on the most important graphs generated by the result of having run the Numerical and the Geometrical algorithms on the 3 data-sets detailed in the last section.

After that I will show the speed tables\footnote{The spread tables can be found in \ref{spread_tables}} for all 3 types of numbers (Sequential, Prime and Random), as well as the combined results for all these data-sets. I will then comment on these results, and try to draw some conclusions. The format for each tables is that I will first have the tables themselves, then mentioned which of the algorithms produced the best results (and how much faster they were than the other solution). If there are individual data points that takes longer than 1 second, then there will there will be 2 version of the above data: One where these extreme cases are included, and one where they are not. For each type of number I will first have the randomized version, then the non-randomized version, and then the combined version.

The complete, machine-readable, data-sets can be found in this projects github repository at \underline{https://github.com/kjmikkel/LonelyRunner}, in the src/data folder.

When a graph in the following contains a hyphenated ``Random'' in its title, it means that while both the Geometrical and the Numerical algorithm has been given the same input, the order of the input for the Numerical algorithm has been randomized. 

\subsection{Interesting Graphs}

\subsubsection{Sequential Numbers}

\graph{Sequential}{500}{Num_5000}{Here it is very clear that there are several configuration in the start where that takes an inordiante amount of time.}
\graph{Sequential}{500}{Geo}{After the Maximum number is greater than 1000, the Numerical algorithm is much faster than the Geometrical}


\graph{Sequential}{1000}{Num_5000}{The same as in figure \ref{Sequential_500_Num_5000} apllies here, only to a greater extent - and here the cutoff is 2000 runners.}
\graph{Sequential}{1000}{Geo}{Again we see that outside of this special zone the Numerical algorithm is far far faster than the Geometrical}

The best explanation I have been able to come up with for cases like figure \ref{Sequential_500_Num_5000} and figure \ref{Sequential_1000_Num_5000} this is that equation \eqaref{eqa:lonelyRunner} does not so much depend on the size of the numbers, but rather on the on the number of the runners, and the interaction of the runners speeds. In believe cases like the two graphs referenced above, the Numerical algorithm is forced to search through a large part of the possible candiates solutions, which means, that the Numerical algorithms run-time of $O(k * n^3)$\footnote{As proved back in Section \ref{proof_num} on p. \pageref{proof_num}} becomes an influence.

\subsubsection{Primes-Random}
%\graph{Primes-Random}{500}{Geo}{}
%\graph{Primes-Random}{1000}{Geo}{}

\subsubsection{Sequential-Random}
\graph{Sequential-Random}{100}{Num}{Please note the very large Error bar, which is not present in figure \ref{Sequential_100_Num}. This is because of randomized order of the input}


%\graph{Sequential-Random}{1000}{Num}{}
%\graph{Sequential-Random}{1000}{Geo}{}
\graph{Sequential-Random}{2000}{Num}{This graph is very interesting because it shows us, that even when the order has been permuted, there is still a high chance of having the Numerical run for a very long time for the first case. In this case it is around 7 hours. It is also very interesting to note that the Error Bar is very small, meaning that the permutation of the order does not guarantee that a fast solution can be found}
%\graph{Sequential-Random}{500}{Num}{}


\subsection{Sequential Results}
\subsubsection{Random Sequential Results}
\begin{tabular}[3]{c|c|c}
 & Geometrical ($\mu$s) speed & Numerical ($\mu$s) speed\\
\hline
Total speeds & 134022145 & 27177923849 \\ 
\hline 
Only speeds from runs & 133791183 & 10536610 \\ 
that took less than 1 second & & \\ 
\hline
Average speed & 6713 & 1361346 \\
\hline
Average speed from runs that & 6707 & 528 \\ 
took less than 1 second & & \\ 
\end{tabular}\\ \\
The Geometrical algorithm is faster by 27043901704 $\mu$s (or 7.51 hours) compared to the Numerical algorithm.\\
The Numerical algorithm is faster by 123254573 $\mu$s (or 2.05 minutes), if we disregard every calculation that took more than 1 second, compared to the Geometrical algorithm.\\
The Geometrical algorithm produced 17 calculations that took more than 1 second, while the Numerical algorithm produced 17, out of 4 data sets (with a total of 19964 data points)\\


\subsubsection{Non-Random Sequential Results}
\begin{table}[bth!]\footnotesize
 \begin{tabular}[3]{c|r|r}
 & Geometrical ($\mu$s) speed & Numerical ($\mu$s) speed\\
\hline
All speeds & 56401388 & 16892350058 \\ 
\hline 
Only speeds from calculations & 56309876 & 3351541 \\ 
that took less than 1 second & & \\ 
\hline
Average speed & 3764 & 1127359 \\
\hline
Average speed from calculations that & 3761 & 223 \\ 
took less than 1 second & & \\ 
\end{tabular}\\ \\
\caption{Sorted Sequential Results\\
The Geometrical algorithm is faster than the Numerical algorithm by 16835948670 $\mu$s (or 4.68 hours).\\
The Numerical algorithm is faster than the Geometrical algorithm by 52958335 $\mu$s (or 52.9 seconds), if we disregard every calculation that took more than 1 second.\\
The Geometrical algorithm produced 15 calculations that took more than 1 second, while the Numerical algorithm produced 15, using 3 data sets (with a total of 14984 data points)\\
}\label{sequential-normal_speedtable}\end{table}

\subsubsection{Combined Sequential Results}
\label{combined_sequential}
\begin{table}[bth!]\footnotesize
 \begin{tabular}[3]{c|r|r}
 & Geometrical ($\mu$s) speed & Numerical ($\mu$s) speed\\
\hline
All speeds & 111675228 & 18270116964 \\ 
\hline 
Only speeds from calculations & 111558301 & 9089200 \\ 
that took less than 1 second & & \\ 
\hline
Average speed & 3726 & 609654 \\
\hline
Average speed from calculations that & 3724 & 303 \\ 
took less than 1 second & & \\ 
\end{tabular}\\ \\
\caption{Complete Sequential Results\\
The Geometrical algorithm is faster than the Numerical algorithm by 18158441736 $\mu$s (or 5.04 hours).\\
The Numerical algorithm is faster than the Geometrical algorithm by 102469101 $\mu$s (or 1.71 minutes), if we disregard every calculation that took more than 1 second.\\
The Geometrical algorithm produced 19 calculations that took more than 1 second, while the Numerical algorithm produced 19, using 6 data sets (with a total of 29968 data points)\\
}\label{sequential_speedtable}\end{table}

\subsection{Random results}
\label{random_results}
\subsubsection{Non-sorted Random Results}
\label{non-sorted_random_results}
\begin{tabular}[3]{c|c|c}
 & Geometrical ($\mu$s) speed & Numerical ($\mu$s) speed\\
\hline
Total speeds & 811572 & 978321 \\ 
\hline 
Only speeds from runs & 796758 & 690972 \\ 
that took less than 1 second & & \\ 
\hline
Average speed & 32 & 39 \\
\hline
Average speed from runs that & 32 & 27 \\ 
took less than 1 second & & \\ 
\end{tabular}\\ \\
The Geometrical algorithm is faster by 166749 $\mu$s (or 0.167 seconds) compared to the Numerical algorithm.\\
The Numerical algorithm is faster by 105786 $\mu$s (or 0.106 seconds), if we disregard every calculation that took more than 1 second, compared to the Geometrical algorithm.\\
The Geometrical algorithm produced 47 calculations that took more than 1 second, while the Numerical algorithm produced 47, out of 5 data sets (with a total of 24924 data points)\\


\subsubsection{Sorted Random Results}
\subsubsection{Combined Results for the Random Numbers}
\begin{table}[bth!]\footnotesize
 \begin{tabular}[3]{c|r|r}
 & Geometrical ($\mu$s) speed & Numerical ($\mu$s) speed\\
\hline
All speeds & 1186653189 & 1862368538 \\ 
\hline 
Only speeds from calculations & 732974269 & 70998170 \\ 
that took less than 1 second & & \\ 
\hline
Average speed & 23805 & 37360 \\
\hline
Average speed from calculations that & 14823 & 1427 \\ 
took less than 1 second & & \\ 
\end{tabular}\\ \\
\caption{Complete  Results\\
The Geometrical algorithm is faster than the Numerical algorithm by 675715349 $\mu$s (or 11.3 minutes).\\
The Numerical algorithm is faster than the Geometrical algorithm by 661976099 $\mu$s (or 11.0 minutes), if we disregard every calculation that took more than 1 second.\\
The Geometrical algorithm produced 401 calculations that took more than 1 second, while the Numerical algorithm produced 102, using 10 data sets (with a total of 49848 data points)\\
}\label{random_speedtable}\end{table}


\subsection{Prime Results}
\label{prime_results}
\subsubsection{Random Prime Results}
\label{random_prime_results}
\begin{table}[bth!]\footnotesize
 \begin{tabular}[3]{c|r|r}
 & Geometrical ($\mu$s) speed & Numerical ($\mu$s) speed\\
\hline
All speeds & 530841683 & 36604376 \\ 
\hline 
Only speeds from calculations & 525234194 & 36604376 \\ 
that took less than 1 second & & \\ 
\hline
Average speed & 78947 & 5443 \\
\hline
Average speed from calculations that & 78136 & 5443 \\ 
took less than 1 second & & \\ 
\end{tabular}\\ \\
\caption{Random Prime Results\\
The Numerical algorithm is faster than the Geometrical algorithm by 494237307 $\mu$s (or 8.22 minutes).\\
The Numerical algorithm is faster than the Geometrical algorithm by 488629818 $\mu$s (or 8.16 minutes), if we disregard every calculation that took more than 1 second.\\
The Geometrical algorithm produced 2 calculations that took more than 1 second using 10 data sets (with a total of 6724 data points)}\label{prime-random_speedtable}\end{table}

\subsubsection{Non-Random Prime Results}
\label{non-random_prime_results}
\begin{table}[bth!]\footnotesize
 \begin{tabular}[3]{c|r|r}
 & Geometrical ($\mu$s) speed & Numerical ($\mu$s) speed\\
\hline
All speeds & 537163442 & 123889442 \\ 
\hline 
Only speeds from calculations & 530814706 & 64701742 \\ 
that took less than 1 second & & \\ 
\hline
Average speed & 79887 & 18424 \\
\hline
Average speed from calculations that & 78978 & 9626 \\ 
took less than 1 second & & \\ 
\end{tabular}\\ \\
\caption{Sorted Prime Results\\
The Numerical algorithm is faster than the Geometrical algorithm by 413274000 $\mu$s (or 6.90 minutes).\\
The Numerical algorithm is faster than the Geometrical algorithm by 466112964 $\mu$s (or 7.74 minutes), if we disregard every calculation that took more than 1 second.\\
The Geometrical algorithm produced 3 calculations that took more than 1 second, while the Numerical algorithm produced 3, using 10 data sets (with a total of 6724 data points)\\
}\label{prime-normal_speedtable}\end{table}


\subsubsection{Combined Results for the Primes Numbers} 
\label{combined_prime}
\begin{table}[bth!]\footnotesize
 \begin{tabular}[3]{c|r|r}
 & Geometrical ($\mu$s) speed & Numerical ($\mu$s) speed\\
\hline
All speeds & 1068005125 & 160493818 \\ 
\hline 
Only speeds from calculations & 1056048900 & 101306118 \\ 
that took less than 1 second & & \\ 
\hline
Average speed & 79417 & 11934 \\
\hline
Average speed from calculations that & 78557 & 7534 \\ 
took less than 1 second & & \\ 
\end{tabular}\\ \\
\caption{Complete Prime Results\\
The Numerical algorithm is faster than the Geometrical algorithm by 907511307 $\mu$s (or 15.1 minutes).\\
The Numerical algorithm is faster than the Geometrical algorithm by 954742782 $\mu$s (or 15.9 minutes), if we disregard every calculation that took more than 1 second.\\
The Geometrical algorithm produced 5 calculations that took more than 1 second, while the Numerical algorithm produced 3, using 20 data sets (with a total of 13448 data points)\\
}\label{prime_speedtable}\end{table}


\subsection{Total Results}
\subsubsection{Random Total Results}

\begin{tabular}[3]{c|c|c}
 & Geometrical ($\mu$s) speed & Numerical ($\mu$s) speed\\
\hline
Total speeds & 665675400 & 27215506546 \\ 
\hline 
Only speeds from runs & 659822135 & 47831958 \\ 
that took less than 1 second & & \\ 
\hline
Average speed & 12897 & 527309 \\
\hline
Average speed from runs that & 12800 & 927 \\ 
took less than 1 second & & \\ 
\end{tabular}\\ \\
The Geometrical algorithm is faster by 26549831146 $\mu$s (or 7.37 hours) compared to the Numerical algorithm.\\
The Numerical algorithm is faster by 611990177 $\mu$s (or 10.2 minutes), if we disregard every calculation that took more than 1 second, compared to the Geometrical algorithm.\\
The Geometrical algorithm produced 66 calculations that took more than 1 second, while the Numerical algorithm produced 64, out of 19 data sets (with a total of 51612 data points)\\


\subsubsection{Non-Random Total Results}

\begin{table}[bth!]\footnotesize
 \begin{tabular}[3]{c|r|r}
 & Geometrical ($\mu$s) speed & Numerical ($\mu$s) speed\\
\hline
All speeds & 1779406447 & 18877629717 \\ 
\hline 
Only speeds from calculations & 1319302093 & 138360481 \\ 
that took less than 1 second & & \\ 
\hline
Average speed & 38158 & 404821 \\
\hline
Average speed from calculations that & 28519 & 2971 \\ 
took less than 1 second & & \\ 
\end{tabular}\\ \\
\caption{Sorted  Results\\
The Geometrical algorithm is faster than the Numerical algorithm by 17098223270 $\mu$s (or 4.75 hours).\\
The Numerical algorithm is faster than the Geometrical algorithm by 1180941612 $\mu$s (or 19.7 minutes), if we disregard every calculation that took more than 1 second.\\
The Geometrical algorithm produced 372 calculations that took more than 1 second, while the Numerical algorithm produced 73, using 18 data sets (with a total of 46632 data points)\\
}\label{total-normal_speedtable}\end{table}

\subsubsection{Combined Total Results}
\label{total_combined_results} 
\begin{tabular}[3]{c|c|c}
 & Geometrical ($\mu$s) speed & Numerical ($\mu$s) speed\\
\hline
Total speeds & 1259240230 & 44231746046 \\ 
\hline 
Only speeds from runs & 1246946717 & 115885241 \\ 
that took less than 1 second & & \\ 
\hline
Average speed & 17174 & 603269 \\
\hline
Average speed from runs that & 17026 & 1582 \\ 
took less than 1 second & & \\ 
\end{tabular}\\ \\
The Geometrical algorithm is faster by 42972505816 $\mu$s (or 11.9 hours) compared to the Numerical algorithm.\\
The Numerical algorithm is faster by 1131061476 $\mu$s (or 18.8 minutes), if we disregard every calculation that took more than 1 second, compared to the Geometrical algorithm.\\
The Geometrical algorithm produced 84 calculations that took more than 1 second, while the Numerical algorithm produced 82, out of 32 data sets (with a total of 73320 data points)\\


\subsection{Analysis}

\subsubsection{Sequential Numbers}
In Section \ref{combined_sequential} it is clear that, not only is the Geometrical algorithm more stable than the Numerical algorithm, but I would also argue that, for this instance, it is the preferable algorithm. Even if we were to use somehow could exclude the extreme cases, we would only save 3 minutes by using the Numerical algorithm. Clearly it is not worth risking about 12 hours of run-time for that. 

  

\subsubsection{Random Numbers}
The result in Section \ref{non-sorted_random_results} This result is actually rather surprising. Since the random numbers tend to be rather large (since they lie between 1 and $2^{32} - 1$), I would have expected the Geometrical solution to be slower, since more Event Points would have had to be created, inserted and removed from the Event Queue, before any of the Runners could enter the Zone. The fact that there are no calculations by the Numerical algorithm that took over 1 second is not that surprising, given that completely random speeds were being used for the tested configurations.\\

\subsubsection{Prime Numbers}

In Section \ref{non-random_prime_results} it is interesting to note that the speed is slightly slower than in Section \ref{random_prime_results}, where the runner speeds had been randomized, clearly showing that the order of the inputs matter.\\

\ref{combined_prime}
Based on the above 2 tables, it seems clear to me that the Prime numbers contain properties that makes the Numerical algorithm more suitable. It is also clear that none of the algorithms is able to find a solution right away, but rather has to work on it. 


\subsubsection{Total Results}
From Section \ref{total_combined_results} it seems clear that using the Geometrical algorithm would be the best choice. In the best case it is much faster than the Numerical algorithm, and in the worst case it only takes about 30 min more than the Numerical algorithm. The risk of having the Numerical algorithm spend hours trying to find a solution for a single configuration makes such small saving irrelevant.  

Another fact worth noting is that while both the Numerical and the Geometrical algorithm have near the same number of calculations that takes more than 1 second, but clearly the time taken by the Numerical ones must take far longer, in for the Numerical solution to be the fastest when these are not counted.



While it from the graphs would seem clear that the Numerical algorithm is faster for most of the instances. 
I believe the reason for this to belong to the fundamental differences of the 2 algorithms. While both algorithms are based on exploring all possible candidates times for a time where equation \eqaref{eqa:lonelyRunner} is true, the Geometrical algorithm works by sweeping through the entire first round of the final runner\footnote{The runner with speed 1}, while the Numerical algorithm often is less methodical about it, depending on the 2 selected runner speeds, trying time points against equation \eqaref{eqa:lonelyRunner}. In practice it is clear this is a better approach than the Geometrical algorithm.
 
It however also clear that for some configurations (even when accounting for the spread), finding a time where equation \eqaref{eqa:lonelyRunner} is true, takes much more time than most of the other configurations for that number of runners. Two good examples of this are \figureref{Primes_1000_Geo} and \figureref{Sequential-Random_1000_Geo}. 

\subsection{Conclusion}

From the above it is clear that the Numerical algorithm is faster than the Geometrical. It is also clear from the graphs that neither of the algorithms are entirely stable, and for some configurations have both a very large execution time spread. As I have not been able to find a pattern for the configurations that have produced some of these abnormalities, I can 
