\section{Results}
\label{results}

\subsection{Introduction}
In this section I will show and discuss the most important results I have found while testing the Geometrical and Numerical algorithms. 

I will first show, and comment, on the most important graphs generated by the result of having run the Numerical and the Geometrical algorithms on the 3 data-sets detailed in the last section.

After that I will show the speed and spread tables for all 3 types of numbers (Sequential, Prime and Random), as well as the combined results for all these data-sets. I will then comment on these results, and try to draw some conclusions. The format for each tables is that I will first have the tables themselves, then mentioned which of the algorithms produced the best results (and how much faster they were than the other solution). If there are individual data points that takes longer than 1 second, then there will there will be 2 version of the above data: One where these extreme cases are included, and one where they are not. For each type of number I will first have the randomized version, then the non-randomized version, and then the combined version.

The complete, machine-readable, data-sets can be found in this projects github repository at \underline{https://github.com/kjmikkel/LonelyRunner}, in the src/data folder.

When a graph in the following contains a hyphenated ``Random'' in its title, it means that while both the Geometrical and the Numerical algorithm has been given the same input, the order of the input for the Numerical algorithm has been randomized. 

\subsection{Interesting Graphs}

\subsubsection{Sequential}
\hide{
\graph{Sequential}{500}{Num_5000}{Here it is very clear that there are several configuration in the start where that takes an inordinate amount of time.}
\graph{Sequential}{500}{Geo}{After the Maximum number is greater than 1000, the Numerical algorithm is much faster than the Geometrical}

\graph{Sequential}{1000}{Num_5000}{The same as in figure \ref{Sequential_500_Num_5000} applies here, only to a greater extent - and here the cutoff is 2000 runners.}
\graph{Sequential}{1000}{Geo}{Again we see that outside of this special zone the Numerical algorithm is far far faster than the Geometrical}
}
The best explanation I have been able to come up with for cases like figure \ref{Sequential_500_Num_5000} and figure \ref{Sequential_1000_Num_5000} this is that equation \eqaref{eqa:lonelyRunner} does not so much depend on the size of the numbers, but rather on the on the number of the runners, and the interaction of the runners speeds. Clearly, in cases like the above, the Numerical algorithm must be forced to search through a large part of the possible candidate solutions, making the Numerical algorithms worst case run-time of $O(k * n^3)$\footnote{As proved back in Section \ref{proof_num} on p. \pageref{proof_num}} visible.

\subsubsection{Sequential-Random}
\hide{
\graph{Sequential-Random}{100}{Num}{Please note the very large Error bar, which is not present in figure \ref{Sequential_100_Num}. This is because of randomized order of the input}
\graph{Sequential-Random}{2000}{Num}{This graph is very interesting because it shows us, that even when the order has been permuted, there is still a high chance of having the Numerical run for a very long time for the first case. In this case it is around 7 hours. It is also very interesting to note that the Error Bar is very small, meaning that the permutation of the order does not guarantee that a fast solution can be found}
}
\subsubsection{Primes-Random}
%\graph{Primes-Random}{500}{Geo}{}
%\graph{Primes-Random}{1000}{Geo}{}



%\graph{Sequential-Random}{1000}{Num}{}
%\graph{Sequential-Random}{1000}{Geo}{}

%\graph{Sequential-Random}{500}{Num}{}

\subsection{Sequential Number Results}

\subsubsection*{Speed}
\FloatBarrier
\label{sorted_sequential}
\begin{table}[bth!]\footnotesize
 \begin{tabular}[3]{c|r|r}
 & Geometrical ($\mu$s) speed & Numerical ($\mu$s) speed\\
\hline
All speeds & 56401388 & 16892350058 \\ 
\hline 
Only speeds from calculations & 56309876 & 3351541 \\ 
that took less than 1 second & & \\ 
\hline
Average speed & 3764 & 1127359 \\
\hline
Average speed from calculations that & 3761 & 223 \\ 
took less than 1 second & & \\ 
\end{tabular}\\ \\
\caption{Sorted Sequential Results\\
The Geometrical algorithm is faster than the Numerical algorithm by 16835948670 $\mu$s (or 4.68 hours).\\
The Numerical algorithm is faster than the Geometrical algorithm by 52958335 $\mu$s (or 52.9 seconds), if we disregard every calculation that took more than 1 second.\\
The Geometrical algorithm produced 15 calculations that took more than 1 second, while the Numerical algorithm produced 15, using 3 data sets (with a total of 14984 data points)\\
}\label{sequential-normal_speedtable}\end{table}
\label{random_sequential}
\begin{tabular}[3]{c|c|c}
 & Geometrical ($\mu$s) speed & Numerical ($\mu$s) speed\\
\hline
Total speeds & 134022145 & 27177923849 \\ 
\hline 
Only speeds from runs & 133791183 & 10536610 \\ 
that took less than 1 second & & \\ 
\hline
Average speed & 6713 & 1361346 \\
\hline
Average speed from runs that & 6707 & 528 \\ 
took less than 1 second & & \\ 
\end{tabular}\\ \\
The Geometrical algorithm is faster by 27043901704 $\mu$s (or 7.51 hours) compared to the Numerical algorithm.\\
The Numerical algorithm is faster by 123254573 $\mu$s (or 2.05 minutes), if we disregard every calculation that took more than 1 second, compared to the Geometrical algorithm.\\
The Geometrical algorithm produced 17 calculations that took more than 1 second, while the Numerical algorithm produced 17, out of 4 data sets (with a total of 19964 data points)\\

\label{combined_sequential}
\begin{table}[bth!]\footnotesize
 \begin{tabular}[3]{c|r|r}
 & Geometrical ($\mu$s) speed & Numerical ($\mu$s) speed\\
\hline
All speeds & 111675228 & 18270116964 \\ 
\hline 
Only speeds from calculations & 111558301 & 9089200 \\ 
that took less than 1 second & & \\ 
\hline
Average speed & 3726 & 609654 \\
\hline
Average speed from calculations that & 3724 & 303 \\ 
took less than 1 second & & \\ 
\end{tabular}\\ \\
\caption{Complete Sequential Results\\
The Geometrical algorithm is faster than the Numerical algorithm by 18158441736 $\mu$s (or 5.04 hours).\\
The Numerical algorithm is faster than the Geometrical algorithm by 102469101 $\mu$s (or 1.71 minutes), if we disregard every calculation that took more than 1 second.\\
The Geometrical algorithm produced 19 calculations that took more than 1 second, while the Numerical algorithm produced 19, using 6 data sets (with a total of 29968 data points)\\
}\label{sequential_speedtable}\end{table}
\FloatBarrier
\subsubsection*{Spread}
\FloatBarrier
\begin{tabular}[3]{c|c|c}
 & Geometrical ($\mu$s) spread & Numerical ($\mu$s) spread\\
\hline
Total spreads & 88221 & 1571476 \\ 
\hline 
Only spreads from runs & 87690 & 41086 \\ 
that took less than 1 second & & \\ 
\hline
Average spread & 5 & 104 \\
\hline
Average spread from runs that & 5 & 2 \\ 
had less spread than 1 second & & \\ 
\end{tabular}\\ \\
The Geometrical algorithm has a lesser spread by 1483255 $\mu$s (or 1.48 seconds) compared to the Numerical algorithm.\\
The Numerical algorithm has a lesser spread by 46604 $\mu$s (or 0.0464 seconds), if we disregard every spread that is larger than 1 second, compared to the Geometrical algorithm.\\
The Geometrical algorithm produced 15 spreads that is larger than 1 second, while the Numerical algorithm produced 15, out of 3 data sets (with a total of 14984 data points)\\

\begin{table}[bth!]\footnotesize
 \begin{tabular}[3]{c|r|r}
 & Geometrical ($\mu$s) spread & Numerical ($\mu$s) spread\\
\hline
All spreads & 285835 & 1141640236 \\ 
\hline 
Only spreads from calculations & 284813 & 840738 \\ 
that took less than 1 second & & \\ 
\hline
Average spread & 19 & 76190 \\
\hline
Average spread from calculations that & 19 & 56 \\ 
had less spread than 1 second & & \\ 
\end{tabular}\\ \\
\caption{Random Sequential Results\\
The Geometrical algorithm has a lesser spread than the Numerical algorithm by 1141354401 $\mu$s (or 19.0 minutes).\\
The Geometrical algorithm has a lesser spread than the Numerical algorithm by 555925 $\mu$s (or 0.554 seconds), if we disregard every spread that is larger than 1 second.\\
The Geometrical algorithm produced 4 spreads that is larger than 1 second, while the Numerical algorithm produced 7, using 3 data sets (with a total of 14984 data points)\\
}\label{sequential-random_spreadtable}\end{table}
\begin{table}[bth!]\footnotesize
 \begin{tabular}[3]{c|r|r}
 & Geometrical ($\mu$s) spread & Numerical ($\mu$s) spread\\
\hline
All spreads & 374056 & 1143211712 \\ 
\hline 
Only spreads from calculations & 372503 & 881824 \\ 
that took less than 1 second & & \\ 
\hline
Average spread & 12 & 38147 \\
\hline
Average spread from calculations that & 12 & 29 \\ 
had less spread than 1 second & & \\ 
\end{tabular}\\ \\
\caption{Complete Sequential Results\\
The Geometrical algorithm has a lesser spread than the Numerical algorithm by 1142837656 $\mu$s (or 19.0 minutes).\\
The Geometrical algorithm has a lesser spread than the Numerical algorithm by 509321 $\mu$s (or 0.508 seconds), if we disregard every spread that is larger than 1 second.\\
The Geometrical algorithm produced 19 spreads that is larger than 1 second, while the Numerical algorithm produced 22, using 6 data sets (with a total of 29968 data points)\\
}\label{sequential_spreadtable}\end{table}
\FloatBarrier
\subsubsection*{Analysis}
Comparing the results found in Section \ref{sorted_sequential} and Section \ref{random_sequential}, we can see a clear example of that randomizing the order of the of the inputs for the Numerical algorithm can give it a better speed.

From Section \ref{combined_sequential} it is clear that that the Geometrical algorithm proves to the be the preferable algorithm. Even if we were to find a way to exclude the extreme cases, we would only save 2 minutes by using the Numerical algorithm. Clearly it not a saving worth risking 5 hours of run-time for. 

Strangely from Section \ref{sorted_sequential} and Section \ref{random_sequential} we can see that the Geometrical algorithm produces a different number of calculations that takes more than 1 second, and in both cases it is the same as the Numerical algorithm, even though the time spent in both cases does not differ that much.

\subsection{Prime Number Results}

\subsubsection*{Speed}
\FloatBarrier
\begin{table}[bth!]\footnotesize
 \begin{tabular}[3]{c|r|r}
 & Geometrical ($\mu$s) speed & Numerical ($\mu$s) speed\\
\hline
All speeds & 537163442 & 123889442 \\ 
\hline 
Only speeds from calculations & 530814706 & 64701742 \\ 
that took less than 1 second & & \\ 
\hline
Average speed & 79887 & 18424 \\
\hline
Average speed from calculations that & 78978 & 9626 \\ 
took less than 1 second & & \\ 
\end{tabular}\\ \\
\caption{Sorted Prime Results\\
The Numerical algorithm is faster than the Geometrical algorithm by 413274000 $\mu$s (or 6.90 minutes).\\
The Numerical algorithm is faster than the Geometrical algorithm by 466112964 $\mu$s (or 7.74 minutes), if we disregard every calculation that took more than 1 second.\\
The Geometrical algorithm produced 3 calculations that took more than 1 second, while the Numerical algorithm produced 3, using 10 data sets (with a total of 6724 data points)\\
}\label{prime-normal_speedtable}\end{table}
\begin{table}[bth!]\footnotesize
 \begin{tabular}[3]{c|r|r}
 & Geometrical ($\mu$s) speed & Numerical ($\mu$s) speed\\
\hline
All speeds & 530841683 & 36604376 \\ 
\hline 
Only speeds from calculations & 525234194 & 36604376 \\ 
that took less than 1 second & & \\ 
\hline
Average speed & 78947 & 5443 \\
\hline
Average speed from calculations that & 78136 & 5443 \\ 
took less than 1 second & & \\ 
\end{tabular}\\ \\
\caption{Random Prime Results\\
The Numerical algorithm is faster than the Geometrical algorithm by 494237307 $\mu$s (or 8.22 minutes).\\
The Numerical algorithm is faster than the Geometrical algorithm by 488629818 $\mu$s (or 8.16 minutes), if we disregard every calculation that took more than 1 second.\\
The Geometrical algorithm produced 2 calculations that took more than 1 second using 10 data sets (with a total of 6724 data points)}\label{prime-random_speedtable}\end{table}
\begin{table}[bth!]\footnotesize
 \begin{tabular}[3]{c|r|r}
 & Geometrical ($\mu$s) speed & Numerical ($\mu$s) speed\\
\hline
All speeds & 1068005125 & 160493818 \\ 
\hline 
Only speeds from calculations & 1056048900 & 101306118 \\ 
that took less than 1 second & & \\ 
\hline
Average speed & 79417 & 11934 \\
\hline
Average speed from calculations that & 78557 & 7534 \\ 
took less than 1 second & & \\ 
\end{tabular}\\ \\
\caption{Complete Prime Results\\
The Numerical algorithm is faster than the Geometrical algorithm by 907511307 $\mu$s (or 15.1 minutes).\\
The Numerical algorithm is faster than the Geometrical algorithm by 954742782 $\mu$s (or 15.9 minutes), if we disregard every calculation that took more than 1 second.\\
The Geometrical algorithm produced 5 calculations that took more than 1 second, while the Numerical algorithm produced 3, using 20 data sets (with a total of 13448 data points)\\
}\label{prime_speedtable}\end{table}
%\label{non-random_prime_results}
%\label{random_prime_results}
%\label{prime_results}
%\label{combined_prime}
\FloatBarrier
\subsubsection*{Spread}
\FloatBarrier
\begin{table}[bth!]\footnotesize
 \begin{tabular}[3]{c|r|r}
 & Geometrical ($\mu$s) spread & Numerical ($\mu$s) spread\\
\hline
All spreads & 1716535 & 1658761 \\ 
\hline 
Only spreads from calculations & 1696751 & 1137956 \\ 
that took less than 1 second & & \\ 
\hline
Average spread & 255 & 246 \\
\hline
Average spread from calculations that & 252 & 169 \\ 
had less spread than 1 second & & \\ 
\end{tabular}\\ \\
\caption{Sorted Prime Results\\
The Numerical algorithm has a lesser spread than the Geometrical algorithm by 57774 $\mu$s (or 0.0576 seconds).\\
The Numerical algorithm has a lesser spread than the Geometrical algorithm by 558795 $\mu$s (or 0.558 seconds), if we disregard every spread that is larger than 1 second.\\
The Geometrical algorithm produced 3 spreads that is larger than 1 second, while the Numerical algorithm produced 3, using 10 data sets (with a total of 6724 data points)\\
}\label{prime-normal_spreadtable}\end{table}
\begin{tabular}[3]{c|c|c}
 & Geometrical ($\mu$s) spread & Numerical ($\mu$s) spread\\
\hline
Total spreads & 656562 & 30502561 \\ 
\hline 
Only spreads from runs & 656562 & 28009439 \\ 
that took less than 1 second & & \\ 
\hline
Average spread & 97 & 4536 \\
\hline
Average spread from runs that & 97 & 4166 \\ 
had less spread than 1 second & & \\ 
\end{tabular}\\ \\
The Geometrical algorithm has a lesser spread by 29845999 $\mu$s (or 29.8 seconds) compared to the Numerical algorithm.\\
The Geometrical algorithm has a lesser spread by 27352877 $\mu$s (or 27.4 seconds), if we disregard every spread that is larger than 1 second, compared to the Numerical algorithm.\\
The Numerical algorithm produced 1 spread that is larger than 1 second out of 10 data sets (with a total of 6724 data points)\\

\begin{tabular}[3]{c|c|c}
 & Geometrical ($\mu$s) spread & Numerical ($\mu$s) spread\\
\hline
Total spreads & 2373097 & 32161322 \\ 
\hline 
Only spreads from runs & 2353313 & 29147395 \\ 
that took less than 1 second & & \\ 
\hline
Average spread & 176 & 2391 \\
\hline
Average spread from runs that & 175 & 2168 \\ 
had less spread than 1 second & & \\ 
\end{tabular}\\ \\
The Geometrical algorithm has a lesser spread by 29788225 $\mu$s (or 29.8 seconds) compared to the Numerical algorithm.\\
The Geometrical algorithm has a lesser spread by 26794082 $\mu$s (or 26.8 seconds), if we disregard every spread that is larger than 1 second, compared to the Numerical algorithm.\\
The Geometrical algorithm produced 3 spreads that is larger than 1 second, while the Numerical algorithm produced 4, out of 20 data sets (with a total of 13448 data points)\\

\FloatBarrier
\subsubsection*{Analysis}
\label{result_prime}
It is interesting to note that Section \ref{prime_results} is the only batch of results where the Numerical algorithm is always faster than the Geometrical algorithm. I can only attribute this to the special nature of the prime numbers, as I have no other evidence. This leads one to wonder whether the Numerical algorithm could be made faster by skipping the second speed if it and the first were not prime (i.e. gcd($speed_1$, $speed_2$) = 1).

Comparing the average speeds in Section \ref{non-random_prime_results} and Section \ref{random_prime_results} we can see that it is faster in the one, where the runner speeds had been randomized, clearly showing that the order of the inputs matter.

\subsection{Random Numbers results}

\subsubsection*{Speed}
\FloatBarrier
\label{random_results}
\label{sorted_random_results}
\begin{table}[bth!]\footnotesize
 \begin{tabular}[3]{c|r|r}
 & Geometrical ($\mu$s) speed & Numerical ($\mu$s) speed\\
\hline
All speeds & 1185841617 & 1861390217 \\ 
\hline 
Only speeds from calculations & 732177511 & 70307198 \\ 
that took less than 1 second & & \\ 
\hline
Average speed & 47578 & 74682 \\
\hline
Average speed from calculations that & 29799 & 2827 \\ 
took less than 1 second & & \\ 
\end{tabular}\\ \\
\caption{Sorted Random Results\\
The Geometrical algorithm is faster than the Numerical algorithm by 675548600 $\mu$s (or 11.3 minutes).\\
The Numerical algorithm is faster than the Geometrical algorithm by 661870313 $\mu$s (or 11.0 minutes), if we disregard every calculation that took more than 1 second.\\
The Geometrical algorithm produced 354 calculations that took more than 1 second, while the Numerical algorithm produced 55, using 5 data sets (with a total of 24924 data points)\\
}\label{random-normal_speedtable}\end{table}
\label{non-sorted_random_results}
\begin{tabular}[3]{c|c|c}
 & Geometrical ($\mu$s) speed & Numerical ($\mu$s) speed\\
\hline
Total speeds & 811572 & 978321 \\ 
\hline 
Only speeds from runs & 796758 & 690972 \\ 
that took less than 1 second & & \\ 
\hline
Average speed & 32 & 39 \\
\hline
Average speed from runs that & 32 & 27 \\ 
took less than 1 second & & \\ 
\end{tabular}\\ \\
The Geometrical algorithm is faster by 166749 $\mu$s (or 0.167 seconds) compared to the Numerical algorithm.\\
The Numerical algorithm is faster by 105786 $\mu$s (or 0.106 seconds), if we disregard every calculation that took more than 1 second, compared to the Geometrical algorithm.\\
The Geometrical algorithm produced 47 calculations that took more than 1 second, while the Numerical algorithm produced 47, out of 5 data sets (with a total of 24924 data points)\\

\label{non-sorted_random_results}
\begin{tabular}[3]{c|c|c}
 & Geometrical ($\mu$s) speed & Numerical ($\mu$s) speed\\
\hline
Total speeds & 811572 & 978321 \\ 
\hline 
Only speeds from runs & 796758 & 690972 \\ 
that took less than 1 second & & \\ 
\hline
Average speed & 32 & 39 \\
\hline
Average speed from runs that & 32 & 27 \\ 
took less than 1 second & & \\ 
\end{tabular}\\ \\
The Geometrical algorithm is faster by 166749 $\mu$s (or 0.167 seconds) compared to the Numerical algorithm.\\
The Numerical algorithm is faster by 105786 $\mu$s (or 0.106 seconds), if we disregard every calculation that took more than 1 second, compared to the Geometrical algorithm.\\
The Geometrical algorithm produced 47 calculations that took more than 1 second, while the Numerical algorithm produced 47, out of 5 data sets (with a total of 24924 data points)\\

\FloatBarrier
\subsubsection*{Spread}
\FloatBarrier
\begin{table}[bth!]\footnotesize
 \begin{tabular}[3]{c|r|r}
 & Geometrical ($\mu$s) spread & Numerical ($\mu$s) spread\\
\hline
All spreads & 2756609 & 1780588 \\ 
\hline 
Only spreads from calculations & 2740190 & 223502 \\ 
that took less than 1 second & & \\ 
\hline
Average spread & 110 & 71 \\
\hline
Average spread from calculations that & 110 & 8 \\ 
had less spread than 1 second & & \\ 
\end{tabular}\\ \\
\caption{Sorted Random Results\\
The Numerical algorithm has a lesser spread than the Geometrical algorithm by 976021 $\mu$s (or 0.978 seconds).\\
The Numerical algorithm has a lesser spread than the Geometrical algorithm by 2516688 $\mu$s (or 2.51 seconds), if we disregard every spread that is larger than 1 second.\\
The Geometrical algorithm produced 55 spreads that is larger than 1 second, while the Numerical algorithm produced 55, using 5 data sets (with a total of 24924 data points)\\
}\label{random-normal_spreadtable}\end{table}
\begin{tabular}[3]{c|c|c}
 & Geometrical ($\mu$s) spread & Numerical ($\mu$s) spread\\
\hline
Total spreads & 1748386012 & 4320649261 \\ 
\hline 
Only spreads from runs & 904878576 & 167413192 \\ 
that took less than 1 second & & \\ 
\hline
Average spread & 70148 & 173352 \\
\hline
Average spread from runs that & 36786 & 6729 \\ 
had less spread than 1 second & & \\ 
\end{tabular}\\ \\
The Geometrical algorithm has a lesser spread by 2572263249 $\mu$s (or 42.9 minutes) compared to the Numerical algorithm.\\
The Numerical algorithm has a lesser spread by 737465384 $\mu$s (or 12.3 minutes), if we disregard every spread that is larger than 1 second, compared to the Geometrical algorithm.\\
The Geometrical algorithm produced 326 spreads that is larger than 1 second, while the Numerical algorithm produced 47, out of 5 data sets (with a total of 24924 data points)\\

\begin{table}[bth!]\footnotesize
 \begin{tabular}[3]{c|r|r}
 & Geometrical ($\mu$s) spread & Numerical ($\mu$s) spread\\
\hline
All spreads & 1751142621 & 4322429849 \\ 
\hline 
Only spreads from calculations & 907618766 & 167636694 \\ 
that took less than 1 second & & \\ 
\hline
Average spread & 35129 & 86712 \\
\hline
Average spread from calculations that & 18347 & 3369 \\ 
had less spread than 1 second & & \\ 
\end{tabular}\\ \\
\caption{Complete  Results\\
The Geometrical algorithm has a lesser spread than the Numerical algorithm by 2571287228 $\mu$s (or 42.8 minutes).\\
The Numerical algorithm has a lesser spread than the Geometrical algorithm by 739982072 $\mu$s (or 12.4 minutes), if we disregard every spread that is larger than 1 second.\\
The Geometrical algorithm produced 381 spreads that is larger than 1 second, while the Numerical algorithm produced 102, using 10 data sets (with a total of 49848 data points)\\
}\label{random_spreadtable}\end{table}
\FloatBarrier
\subsubsection*{Analysis}
The result in Section \ref{non-sorted_random_results} This result is actually rather surprising. Since the random numbers tend to be rather large (since they lie between 1 and $2^{32} - 1$), I would have expected the Geometrical solution to be slower, since more Event Points would have had to be created, inserted and removed from the Event Queue, before any of the Runners could enter the Zone. The fact that there are no calculations by the Numerical algorithm that took over 1 second is not that surprising, given that completely random speeds were being used for the tested configurations.\\



\subsection{Total Results}
\subsubsection*{Speed}
\FloatBarrier
\begin{table}[bth!]\footnotesize
 \begin{tabular}[3]{c|r|r}
 & Geometrical ($\mu$s) speed & Numerical ($\mu$s) speed\\
\hline
All speeds & 1779406447 & 18877629717 \\ 
\hline 
Only speeds from calculations & 1319302093 & 138360481 \\ 
that took less than 1 second & & \\ 
\hline
Average speed & 38158 & 404821 \\
\hline
Average speed from calculations that & 28519 & 2971 \\ 
took less than 1 second & & \\ 
\end{tabular}\\ \\
\caption{Sorted  Results\\
The Geometrical algorithm is faster than the Numerical algorithm by 17098223270 $\mu$s (or 4.75 hours).\\
The Numerical algorithm is faster than the Geometrical algorithm by 1180941612 $\mu$s (or 19.7 minutes), if we disregard every calculation that took more than 1 second.\\
The Geometrical algorithm produced 372 calculations that took more than 1 second, while the Numerical algorithm produced 73, using 18 data sets (with a total of 46632 data points)\\
}\label{total-normal_speedtable}\end{table}
\begin{tabular}[3]{c|c|c}
 & Geometrical ($\mu$s) speed & Numerical ($\mu$s) speed\\
\hline
Total speeds & 665675400 & 27215506546 \\ 
\hline 
Only speeds from runs & 659822135 & 47831958 \\ 
that took less than 1 second & & \\ 
\hline
Average speed & 12897 & 527309 \\
\hline
Average speed from runs that & 12800 & 927 \\ 
took less than 1 second & & \\ 
\end{tabular}\\ \\
The Geometrical algorithm is faster by 26549831146 $\mu$s (or 7.37 hours) compared to the Numerical algorithm.\\
The Numerical algorithm is faster by 611990177 $\mu$s (or 10.2 minutes), if we disregard every calculation that took more than 1 second, compared to the Geometrical algorithm.\\
The Geometrical algorithm produced 66 calculations that took more than 1 second, while the Numerical algorithm produced 64, out of 19 data sets (with a total of 51612 data points)\\

\label{total_combined_results} 
\begin{tabular}[3]{c|c|c}
 & Geometrical ($\mu$s) speed & Numerical ($\mu$s) speed\\
\hline
Total speeds & 1259240230 & 44231746046 \\ 
\hline 
Only speeds from runs & 1246946717 & 115885241 \\ 
that took less than 1 second & & \\ 
\hline
Average speed & 17174 & 603269 \\
\hline
Average speed from runs that & 17026 & 1582 \\ 
took less than 1 second & & \\ 
\end{tabular}\\ \\
The Geometrical algorithm is faster by 42972505816 $\mu$s (or 11.9 hours) compared to the Numerical algorithm.\\
The Numerical algorithm is faster by 1131061476 $\mu$s (or 18.8 minutes), if we disregard every calculation that took more than 1 second, compared to the Geometrical algorithm.\\
The Geometrical algorithm produced 84 calculations that took more than 1 second, while the Numerical algorithm produced 82, out of 32 data sets (with a total of 73320 data points)\\

\FloatBarrier
\subsubsection*{Spread}
\FloatBarrier
\begin{tabular}[3]{c|c|c}
 & Geometrical ($\mu$s) spread & Numerical ($\mu$s) spread\\
\hline
Total spreads & 1749581041 & 6658464713 \\ 
\hline 
Only spreads from runs & 906071908 & 196306070 \\ 
that took less than 1 second & & \\ 
\hline
Average spread & 33898 & 129010 \\
\hline
Average spread from runs that & 17672 & 3808 \\ 
had less spread than 1 second & & \\ 
\end{tabular}\\ \\
The Geometrical algorithm has a lesser spread by 4908883672 $\mu$s (or 1.36 hours) compared to the Numerical algorithm.\\
The Numerical algorithm has a lesser spread by 709765838 $\mu$s (or 11.8 minutes), if we disregard every spread that is larger than 1 second, compared to the Geometrical algorithm.\\
The Geometrical algorithm produced 343 spreads that is larger than 1 second, while the Numerical algorithm produced 69, out of 19 data sets (with a total of 51612 data points)\\

\begin{tabular}[3]{c|c|c}
 & Geometrical ($\mu$s) spread & Numerical ($\mu$s) spread\\
\hline
Total spreads & 1804756 & 3230237 \\ 
\hline 
Only spreads from runs & 1784441 & 1179042 \\ 
that took less than 1 second & & \\ 
\hline
Average spread & 83 & 148 \\
\hline
Average spread from runs that & 82 & 54 \\ 
had less spread than 1 second & & \\ 
\end{tabular}\\ \\
The Geometrical algorithm has a lesser spread by 1425481 $\mu$s (or 1.43 seconds) compared to the Numerical algorithm.\\
The Numerical algorithm has a lesser spread by 605399 $\mu$s (or 0.606 seconds), if we disregard every spread that is larger than 1 second, compared to the Geometrical algorithm.\\
The Geometrical algorithm produced 18 spreads that is larger than 1 second, while the Numerical algorithm produced 18, out of 13 data sets (with a total of 21708 data points)\\

\begin{table}[bth!]\footnotesize
 \begin{tabular}[3]{c|r|r}
 & Geometrical ($\mu$s) spread & Numerical ($\mu$s) spread\\
\hline
All spreads & 1753889774 & 5497802883 \\ 
\hline 
Only spreads from calculations & 910344582 & 197665913 \\ 
that took less than 1 second & & \\ 
\hline
Average spread & 18805 & 58948 \\
\hline
Average spread from calculations that & 9803 & 2122 \\ 
had less spread than 1 second & & \\ 
\end{tabular}\\ \\
\caption{Complete  Results\\
The Geometrical algorithm has a lesser spread than the Numerical algorithm by 3743913109 $\mu$s (or 1.04 hours).\\
The Numerical algorithm has a lesser spread than the Geometrical algorithm by 712678669 $\mu$s (or 11.9 minutes), if we disregard every spread that is larger than 1 second.\\
The Geometrical algorithm produced 403 spreads that is larger than 1 second, while the Numerical algorithm produced 128, using 36 data sets (with a total of 93264 data points)\\
}\label{total_spreadtable}\end{table}
\FloatBarrier
\subsubsection*{Analysis}
From Section \ref{total_combined_results} it seems clear that using the Geometrical algorithm would be the best choice. In the best case it is much faster than the Numerical algorithm, and in the worst case it only takes about 30 min more than the Numerical algorithm. The risk of having the Numerical algorithm spend hours trying to find a solution for a single configuration makes such small saving irrelevant.  

Another fact worth noting is that while both the Numerical and the Geometrical algorithm have near the same number of calculations that takes more than 1 second, but clearly the time taken by the Numerical ones must take far longer, in for the Numerical solution to be the fastest when these are not counted.

\subsection{Conclusion}

From the above it is clear that the Numerical algorithm is faster than the Geometrical. It is also clear from the graphs that neither of the algorithms are entirely stable, and for some configurations have both a very large execution time spread. As I have not been able to find a pattern for the configurations that have produced some of these abnormalities, I can 
