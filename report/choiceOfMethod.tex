\section{Possible methods}
\label{choiceOfMethod}

\subsection{Introduction}
I will in this subsection describe the different methods I have thought of using in this project, and discuss their merits and flaws.

\begin{description}
\item[Computational Geometry:] The intuitive description of the Lonely Runner conjecture (descibed in the Introduction) lends itself well to a geometrical interpretation, and leads me to wonder whether an algorithm based on this fact would be efficient.

\item[Number theoretic:] The Lonely Runner conjecture is, in its original formulation, a problem from number theory. It is therefore not reasonable to assume that a number theoretic approach could lead to great speedups in calculating a possible solution.
\end{description}

\subsection{Computational Geometry}
The following solution is based on the first intiutive description of the conjecture. It is clear that we are interested in the time interval when the runners are in the \zone. However, we are interested in the time they are in the \zone - i.e. when a runner with a speed $s$ enters and leaves the \zone.\\ 

Let $s$ be the speed and $t$ be the time, then from \ref{mainProb} we have 

\begin{equation}
\label{speedOne}
\begin{split}
s * t &= \frac{1}{n+1} \\
t &= \frac{1}{s * (n+1)}
\end{split}
\end{equation}

and 

\begin{equation}
\label{speedTwo}
\begin{split}
s * t &= \frac{n}{n+1} \\
t &= \frac{n}{s * (n+1)}
\end{split}
\end{equation}

From \ref{speedOne} and \ref{speedOne} it is clear that the time interval for the first the runner is in the \zone will be 
\begin{displaymath}
\left[\frac{1}{s * (n+1)}, \frac{n}{s * (n+1)}\right]
\end{displaymath}

More generally, a given runner with speed $s$ will be in the \zone at time 

\begin{displaymath}
\left[\frac{1 + k * (n+1)}{s * (n+1)}, \frac{n + k * (n+1)}{s * (n+1)}\right] 
\end{displaymath}

So now it becomes a matter find out whether there is a time where the time intervals for all the runners overlap. To do this I propose using a horizontal Plane Sweep algorithm. Since the Lonely Runner Conjecture has not been proven, we have to make sure that the plane sweep algorithm will always terminate, but always find a valid solution to the conjecture, if its exists.

My solution to this problem would be to terminate the algorithm is to   

[description about intervals and finnish line - also introduce and the need for a heap]

\subsubsection{Points}

In order to distingush these 3 cases we need to introduce 3 different points:
\begin{description}
\item[\comStart] The first point where the runner is $\frac{1}{n + 1}$ units away from the start line. In the following algorithm the \comStart will contain the time it took to reach that location, and which runner it belongs to.
\item[\comEnd] The first point after the \comStart where the runner is $\frac{1}{n + 1}$ units away from the start line. In the following algorithm \comEnd has the responcibility of finding the next interval its runner is at least $\frac{1}{n+1}$ units away from the startline. I place this responcibility on \comEnd, on the grounds it is the first point where we know the current interval is not going to give a solution to the Lonely Runner conjecture. \comEnd needs to know the time its runner will pass it, the amount of time used to pass $\frac{1}{n+1}$ units, and to which runner it belongs to.
\item[\comFin] The time at which the runner passes the startline. As explainted above \todo{explain it above}, this is needed in order to ensure that the algorithm will always terminate, even if there is no solution to the Lonely Runner in that instance.
\end{description}

\subsubsection{Order of points}
In order to ensure this approch will work, I introduce the order $\prec$. For the timepoints $p$ and $q$, $p \prec q$ iff \\

\begin{center}
$p_{time} < q_{time}$\\
or \\
$p_{time} == q_{time}$ and type($p$) == \comFin and type($q$) != type($p$)\\
or \\
$p_{time} == q_{time}$ and type($p$) == \comStart and type($q$) == \comEnd \\
or \\
$p_{time} == q_{time}$ and type($p$) == type($q$) and $p_{runner} < q_{runner}$
\end{center}

Using this order there is no ambiguity where to place the points, and we also avoid the degernate case where we have (n-1) runners are in the \zone, and the next two points are a \comEnd and a \comStart, both at the same time. Since the \zone is a closed set, it is clear that this instance should return a valid solution to the conjecture, but if \comEnd comes first, then the case would not be reported - therefore \comStart must be placed before \comEnd.

\subsubsection{Algorithm}
\begin{algorithm}[H]
\caption{MakeTimePoints}
\highlights
\SetKwData{start}{startTime}
\Input{A start-time \ti and the \unit used to run $\frac{1}{(\n + 1)}$ part of the track, the number \run of the runner, and the time queue \li}
\Output{The time queue \li, with a new \startT, \eT and \finish inserted for \run runner}
 
Make new \startT $start$ from \start as its start-time, and \run as its runner 
  
Make new \eT $end$ from \start + \unit * \n as its start-time, \unit as its speed, and \run as its runner
  
Make new \finish $finish$ with \start + \unit * (n+1) as its time, and \run as its runner
  
Add $start$, $end$ and $line$ to \li

\return \li
\end{algorithm}

\begin{algorithm}[H]
  \caption{FindLonelyRunnerTime}
  \highlights
  \Input{A list \s which contains n pairwise different speeds for n runners}
  \Output{A time \ti where all n runners are at least $\frac{1}{n + 1}$ units away from the starting line, or a \no, indicating that no such time exists}
  
  Create and initilize \li
  
\End $\gets 0$

\inter $\gets 0$

\n $\gets$ size(\s)

$runnerNum \gets 1$

\ForEach{speed $\in$ \s}{

  \unit $\gets \frac{1}{speed * (\n + 1)}$  

  \li $\gets$ MakeTimePoints(\unit, \unit, $runnerNum$, \li)
  
  $runnerNum += 1$
}
\While{\li is not empty}{
  \p $\gets$ firstPoint(\li)
  
  \If{\p is \startT}{
      \End = 0
      
      \inter $+= 1$
      
      \If{\inter == \n}{
        \return \p.time
      }
    }

  \If{\p is \eT}{
    \End = 0
    
    \inter $-= 1$

    \ti $\gets \frac{2}{\p.\unit * (\n + 1)} + \p.\ti$

    \li $\gets$ MakeTimePoints(\ti, p.\unit, p.\run, \li)
 }
  
  \If{\p is \finish}{
      \End += 1
  
      \If{\End == \n}{
        
        \return \no
        
      }
  }
}
\end{algorithm}

\subsection{Number Theoretic}

