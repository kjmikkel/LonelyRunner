\section{Possible methods}
\label{choiceOfMethod}

\subsection{Introduction}
I will in this subsection describe the different methods I have thought of using in this project, and discuss their merits and flaws.

\begin{description}
\item[Computational Geometry:] The intuitive description of the Lonely Runner conjecture (described in the Introduction) lends itself well to a geometrical interpretation, and leads me to wonder whether an algorithm based on this fact would be efficient.

\item[Number theoretic:] The Lonely Runner conjecture is, in its original formulation, a problem from number theory. It is therefore not reasonable to assume that a number theoretic approach could lead to great speedups in calculating a possible solution.
\end{description}

\subsection{Computational Geometry}

The following solution is based on the first intiutive description of the conjecture. It is clear that we are interested in the time interval when the runners are in the \zone. However, we are interested in the time they are in the \zone - i.e. when a runner with a speed $s$ enters and leaves the \zone.\\ 

Let $s$ be the speed and $t$ be the time, then from \ref{mainProb} we have 

\begin{equation}
\label{speedOne}
\begin{split}
s * t &= \frac{1}{n+1} \\
t &= \frac{1}{s * (n+1)}
\end{split}
\end{equation}

and 

\begin{equation}
\label{speedTwo}
\begin{split}
s * t &= \frac{n}{n+1} \\
t &= \frac{n}{s * (n+1)}
\end{split}
\end{equation}

From \ref{speedOne} and \ref{speedOne} it is clear that the time interval for the first the runner is in the \zone will be 
\begin{displaymath}
\left[\frac{1}{s * (n+1)}, \frac{n}{s * (n+1)}\right]
\end{displaymath}

More generally, a given runner with speed $s$ will be in the \zone at time 

\begin{displaymath}
\left[\frac{1 + k * (n+1)}{s * (n+1)}, \frac{n + k * (n+1)}{s * (n+1)}\right] 
\end{displaymath}

So now it becomes a matter find out whether there is a time where the time intervals for all the runners overlap. To do this I propose using a horizontal Plane Sweep algorithm. Since the Lonely Runner Conjecture has not been proven, we have to make sure that the plane sweep algorithm will always terminate, but always find a valid solution to the conjecture, if its exists.

My solution to this problem would be to terminate the algorithm if all the runners crosses the start-line at the same time - since everything after that will only be a repitition of what had already happened.

\subsubsection{Points}

[description about intervals and finish line]

In order to distinguish these 3 cases we need to introduce 3 different points:
\begin{description}
\item[\comStart] The first point where the runner is $\frac{1}{n + 1}$ units away from the start line. In the following algorithm the \comStart will contain the time it took to reach that location, and which runner it belongs to.
\item[\comEnd] The first point after the \comStart where the runner is $\frac{1}{n + 1}$ units away from the start line. In the following algorithm \comEnd has the responsibility of finding the next interval its runner is at least $\frac{1}{n+1}$ units away from the start-line. I place this responsibility on \comEnd, on the grounds it is the first point where we know the current interval is not going to give a solution to the Lonely Runner conjecture. \comEnd needs to know the time its runner will pass it, the amount of time used to pass $\frac{1}{n+1}$ units, and to which runner it belongs to.
\item[\comFin] The time at which the runner passes the start-line. As explained above \todo{explain it above}, this is needed in order to ensure that the algorithm will always terminate, even if there is no solution to the Lonely Runner in that instance.
\end{description}

\subsubsection{Order of points}
In order to ensure this approch will work, I introduce the order $\prec$. For the timepoints $p$ and $q$, $p \prec q$ iff \\

\begin{center}
$p_{time} < q_{time}$\\
or \\
$p_{time} == q_{time}$ and type($p$) == \comFin and type($q$) != type($p$)\\
or \\
$p_{time} == q_{time}$ and type($p$) == \comStart and type($q$) == \comEnd \\
or \\
$p_{time} == q_{time}$ and type($p$) == type($q$) and $p_{runner} < q_{runner}$
\end{center}

Using this order there is no ambiguity where to place the points, and we also avoid the degernate case where we have (n-1) runners are in the \zone, and the next two points are a \comEnd and a \comStart, both at the same time. Since the \zone is a closed set, it is clear that this instance should return a valid solution to the conjecture, but if \comEnd comes first, then the case would not be reported - therefore \comStart must be placed before \comEnd.

\subsubsection{Algorithm}
\begin{algorithm}[H]
\caption{MakeTimePoints}
\highlights
\SetKwData{start}{startTime}
\Input{A start-time \ti and the \unit used to run $\frac{1}{(\n + 1)}$ part of the track, the number \run of the runner, and the time queue \li}
\Output{The time queue \li, with a new \startT, \eT and \finish inserted for \run runner}
 
Make new \startT $start$ from \start as its start-time, and \run as its runner
  
Make new \eT $end$ from \start + \unit * \n as its start-time, \unit as its speed, and \run as its runner
  
Make new \finish $finish$ with \start + \unit * (n+1) as its time, and \run as its runner
  
Add $start$, $end$ and $finish$ to \li

\return \li
\end{algorithm}

\begin{algorithm}[H]
  \caption{FindLonelyRunnerTime}
  \highlights
  \SetKwData{rTime}{finishTime}
  \SetKwData{lFinish}{LastRunnerToFinish}
  \SetKwData{and}{and}
  \Input{A list \s which contains n pairwise different speeds for n runners}
  \Output{A time \ti where all n runners are at least $\frac{1}{n + 1}$ units away from the starting line, or a \no, indicating that no such time exists}
  
  Create and initialise \li
  
\End $\gets 0$

\inter $\gets 0$

\rTime $\gets -1$

\lFinish $\gets -1$

\n $\gets$ size(\s)

$runnerNum \gets 1$

\ForEach{speed $\in$ \s}{
  
  \unit $\gets \frac{1}{speed * (\n + 1)}$  
  
  \li $\gets$ MakeTimePoints(\unit, \unit, $runnerNum$, \li)
  
  $runnerNum += 1$
}

\While{\li is not empty}{
  \p $\gets$ firstPoint(\li)
  
  \If{\p is \startT}{
      \End = 0
      
      \inter $+= 1$
      
      \If{\inter == \n}{
        \return $\p_{time}$
      }
  }
  
  \If{\p is \eT}{
    \End = 0
    
    \inter $-= 1$
    
    \ti $\gets \frac{2}{\p_{\unit} * (\n + 1)} + \p.\ti$
    
    \li $\gets$ MakeTimePoints(\ti, $p_{\unit}$, $p_{\run}$, \li)
  }
  
  \If{\p is \finish}{
    \If{\rTime == $-1$}{
      
      \rTime $\gets \p_{time}$ 
      
    }
    
    \If{$\p_{time}$ == \rTime \and $\p_{\run}$ == \lFinish}{
   
     \End $+= 1$\; 
     
     \If{\End == \n}{  
       \return \no
     }
     
     
    } \uElse{
      
      \rTime $\gets -1$
   
      \End $\gets 0$       
    }
    
  }
}
\end{algorithm}

\subsubsection{Proofs of the Geometric algorithm}
\algProof{
?
}{
It is clear that the algorithm terminates iff the Lonely Runner conjecture holds or if there is a time $t$ where all $n$ runners hits the starting line.

If the Lonely Runner conjecture holds, then the algorithm will eventually find an intersection of the intervals, and return it, as per the proof above. If the Lonely Runner conjecture does not hold, then there still might be a solution to the given instance, which will be reported. If the Lonely Runner conjecture does not hold, and there is not a solution to the given instance, then the algorithm can only terminate if there is a time $t$ where all the $n$ runners hit the starting line.  

It is clear that a runner with speed $s$ takes $\frac{1}{s}$ time units to complete a single lap around the track, and so in order for all the runners to arrive at starting line at the same time, I need to find a time $t$ that all the $n$ runner's speeds divide. This is obviously true for the time $t = s_1 s_2 \ldots s_n$, so such a time clearly exists. If not all of the runner's speeds are co-prime, then $t$ can be smaller. 
}{
In the worst case the Lonely Runner Conjecture does not hold, and all the speeds of the runner's are co-prime. In that case the algorithm will only terminate after having made the runner's 
}

\subsection{Number Theory}

A Number Theory alternative to the Geometric Algorithm can be made by applying the results of \cite{invs}. I have sketched the algorithm described from \cite{invs} below\todo{Make sure you have the correct algorithm} .

\begin{algorithm}[H]
  \caption{NumericalLonelyRunner}
  \highlights
  \Input{A list \s which contains n pairwise different speeds for n runners}
  \Output{A time \ti such that all n runners are at least $\frac{1}{n + 1}$ units away from the starting line, or a \no, indicating that no such time exists}
  
  $n \gets size(\s)$

  \ForEach{$d \in \s$}{
    
    \ForEach{$d\prime \in \s \setminus \E{d}$}{
      
      $k \gets d + d\prime$
      
      $m \gets \lceil \frac{k}{n + 1} \rceil $
      
      \For{$a \leftarrow m$ \KwTo $n$}{
        
        $testValid \gets true$
        
        \ForEach{$s \in \s$}{
          
          $testValid = testValid \and (\Vert \frac{a}{k} * s \Vert \geq \frac{1}{n+1} )$
          
        }
        
        \If{testValid}{
          
          \return $\frac{a}{k}$
          
        }
      }
    }
  }
  \return \no
\end{algorithm}

\subsubsection{Proofs for the numerical algorithm}

\algProof{
In \cite{invs} the authors prove that if the Lonely Runner conjecture holds, then the algorithm will produce a solution. To take care of the possible case that the Lonely Runner conjecture does not hold, I have added the last return statement, ensuring that the correct answer will always be reported.
}{
It is clear that the algorithm will always terminate, $\Vert x \Vert $ is the only function call in the algorithm, and it will always terminate.
}{
Let us first analyse how many [Something about k and how it influences the runtime]

Let $n$ be the number of runners, then the algorithm runs in $O(n^4)$ time, since the algorithm uses 4 nested For loops, and in the worst case each loop will run through all the runners.
}
