\section{Possible methods}
\label{choiceOfMethod}

\subsection{Introduction}
I will in this subsection describe the different methods I have thought of using in this project, and discuss their merits and flaws.
Possible methods:

\begin{description}
\item[Computational Geometry:] The intuitive description of the Lonely Runner conjecture (\nref{introduction}) lends itself well to a geometrical interpretation, and makes me wonder  leads me to wonder whether an algorithm based on this would be efficient.

\item[Number theoretic:] The Lonely Runner conjecture is, in its original formulation, a problem from number theory. It is therefore not reasonable to assume that a number theoretic approach could lead to an efficient verification program.
\end{description}

\subsection{Computational Geometry}
\label{compGeo}

The following solution is based on the first intuitive description of the conjecture, found in \eqaref{eqa:lonelyRunner}. It is clear that we are interested in the time interval when the runners are in the Zone, but not particularly where they are - more specifically we are interested when a runner enters and leaves the Zone. 

Let $s$ be the speed of a given runner then the runner enters the Zone for the first time when he is $\frac{1}{n+1}$ units away form the start line, so: 

\begin{equation}
\label{eqa:speedOne}
\begin{split}
s * t &= \frac{1}{n+1} \\
t &= \frac{1}{s * (n+1)}
\end{split}
\end{equation}

and leaving the Zone, the runner has passed $\frac{n}{n+1}$ units in the Zone:

\begin{equation}
\label{eqa:speedTwo}
\begin{split}
s * t &= \frac{n}{n+1} \\
t &= \frac{n}{s * (n+1)}
\end{split}
\end{equation}

From \eqaref{eqa:speedOne} and \eqaref{eqa:speedTwo} it is clear that the time interval for the first the runner is in the Zone will be 
\begin{displaymath}
\left[\frac{1}{s * (n+1)}, \frac{n}{s * (n+1)}\right]
\end{displaymath}

More generally, a runner with speed $s$ will be in the Zone for the $k$'th time, where $k \in \N \cup \E{0}$, in the time interval 

\begin{displaymath}
\left[\frac{1 + k * (n+1)}{s * (n+1)}, \frac{n + k * (n+1)}{s * (n+1)}\right] 
\end{displaymath}

Now it becomes a matter find out whether there is a time where the time intervals for all the runners overlap. To do this I propose using a horizontal Plane Sweep algorithm\footnote{For a general overview of Plane Sweep algorithms, please consult \cite{citeulike:3347056} or other university level or higher book dealing with Computational Geometry}, where the interesting cases are when a runner enters, and leaves the Zone. 

However, since the Lonely Runner conjecture has not been proved, there must be a means to terminate the algorithm if for a given configuration of runners there exists a time that make equation \eqaref{eqa:lonelyRunner} true.

My solution to this is to introduce a fake runner. This fake runner has a speed of $1$, and will only serve as a marker for when to terminate the algorithm. The intuition behind this is that based on equation \eqaref{eqa:lonelyRunner} - it is clear that if a solution exist it must do so for $x \in [0,1)$, and therefore if we reach the time $1$, no solution to equation \eqaref{eqa:lonelyRunner} exists. \todo{needs proof}

\subsubsection{Event Points}
\label{eventPoints}
In this section I will discuss the role of Event Points in the Plane Sweep algorithm, as well as define them. As stated in section \ref{compGeo} we are only interested in whether or not a runner is in the Zone, not his exact position on the track. Therefore i introduce the Event Points, which are used to indicate whether a runner is entering a Zone or leaving it, ignoring everything else that might happen for the runner --- say, passing the finishing line. 

One question that must be answered before we go any further, is whether or not we will create all Event Points before the main part of the algorithm begins (if indeed this is possible) or create some of them and generate the remaining be generated at run-time. This also has implications whether a Linked-List or a heap, respectively, should be used for the Priority Queue that will contain the Event Points. Since I have already established that we do not need to produce any Event Points that happen after time 1, it is clear that we can create all the Event Points before entering the main loop.\\ 

\adDis{
Creating Event points before entering the main loop
}{
\item The advantage of creating the points in advance would be that the generation of all the Event Points themselves are an task that can be made in parallel. After this the main loop could simply be run without having to dedicate any resources to creating new events.

\item It would also make the Priority Queue simpler, since instead of a heap, we could simply use a Linked List structure. 
}{
\item Since all the points are generated at the beginning, this would sharply increase the memory footprint on the system.

\item While the Event Points themselves could be generated in parallel, they would still have to be inserted into the Priority Queue, which would serve as a bottleneck. Even if this was done while the other Event Points were still being generated, it would slow down the procedure enormously, loosing a lot of time. The Priority Queue would still have to be sorted, which would take $O(k\log(k))$ time, where $k$ is the sum of all the runners speeds, though of course a  parallel sorting algorithm could be used.

\item If done correctly it would add quite a bit of complexity to the program
}{
Given that generating an Event Point is a trivial operation (it being a very simple data structure), one can question the worth of this implementation.
}

\adDis{
Creating event points on the fly
}{
\item It would reduce the memory footprint drastically, since new event points would only be made when needed.

\item Adding the new event points sequentially would also be much simpler than in the parallel version, since there would be no need for locks to protect the Priority Queue needed.  
}{
\item Creating every Event Point sequentially could potentially take a lot more time than the parallel solution.
}{
Creating the event points as needed is far simpler than the parallel version, although it is potentially slower in situations where the speed of the runners is large, or where the solution to equation \eqaref{eqa:lonelyRunner} is close to $1$, or no solution to the equation exist for the given configuration of runners. 
}

Thus I feel justified in saying that creating the event points on the fly is the preferred solution, as any possible gains by creating all the points at once would be minimal at best\footnote{It is worth noting that the insertions and removals from the Priority Queue cannot be used as an argument either way, since the parallel and sequential version each uses a different data structure, optimised for their operation.}, and would unduly increase the complexity of the program I would have to create. 

In order to distinguish entering and leaving the Zone, as well as the ``fake'' runner I introduce 3 different points:
\begin{description}
\item[\comStart] The first point represents the time where the runner is $\frac{1}{n + 1}$ units away from the start line. In the following algorithm the \comStart\, will contain the time it took to reach that location, and which runner it belongs to.
\item[\comEnd] The first point after the \comStart\, where the runner is $\frac{1}{n + 1}$ units away from the start line. Since \comEnd\, is the last event point for any given runner in the Priority Queue, \comEnd\, is used to signal to the algorithm that a new \comStart\, and \comEnd\, needs to be added to the Priority Queue. \comEnd\, needs to know the time its runner will pass it, the amount of time its runner has to pass $\frac{1}{n+1}$ units, and to which runner it belongs to.
\item[\comFin] The final point in that will be processed --- its time is always set to $1$. If this point is ever reached, then there exists no solution for the given configuration of runners and speeds, and the Lonely Runner conjecture does not hold, for the given configuration.
\end{description}

\subsubsection{Order of points}
In order to avoid any ambiguity concerning the order of the points, I will now introduce an order $\prec$. One of the degenerate cases we need to avoid is when we have (n-1) runners that are in the Zone, and the next two points are a \comEnd\, and a \comStart, both at the same time. Since the Zone is a closed set, it is clear that this instance should return a valid solution, but if \comEnd\, comes first, then the solution would not be reported - therefore \comStart\, must be placed before \comEnd. 

For the time points $p$ and $q$, $p \prec q$ iff \\

\begin{center}
$p_{time} < q_{time}$\\
or \\
$p_{time} == q_{time}$ and type($q$) == \comFin\\
or \\
$p_{time} == q_{time}$ and type($p$) == \comStart and type($q$) == \comEnd \\
or \\
$p_{time} == q_{time}$ and type($p$) == type($q$) and $p_{runner} < q_{runner}$
\end{center}

\subsubsection{Algorithm}
\begin{algorithm}[H]
\caption{MakeTimePoints}
\highlights
\SetKwData{start}{startTime}
\Input{A start-time \ti and the \unit used to run $\frac{1}{(\n + 1)}$ part of the track, the number \run of the runner, and the time queue \li}
\Output{The time queue \li, with a new \startT and \eT inserted for \run as its runner}
 
Make new \startT $start$ from \start as its start-time, and \run as its runner
  
Make new \eT $end$ from \start + \unit * \n as its start-time, \unit as its speed, and \run as its runner
    
Add $start$ and $end$ to \li

\return \li
\end{algorithm}

\begin{algorithm}[H]
  \caption{FindLonelyRunnerTime}
  \highlights
  \SetKwData{and}{and}
  \Input{A list \s which contains n pairwise different speeds for n runners}
  \Output{A time \ti where all n runners are at least $\frac{1}{n + 1}$ units away from the starting line, or a \no, indicating that no such time exists}

  Create and initialise \li
  
  \inter $\gets 0$
  
  \n $\gets$ size(\s)
  
  $runnerNum \gets 1$
  
  Make \finish with time $1$ and add it to \li

  \ForEach{speed $\in$ \s}{
    
    \unit $\gets \frac{1}{speed * (\n + 1)}$  
    
    \li $\gets$ MakeTimePoints(\unit, \unit, $runnerNum$, \li)
    
    $runnerNum += 1$
  }
  
  \While{\li is not empty}{

    \p $\gets$ firstPoint(\li)
    
    \If{\p is \startT}{      
      \inter $+= 1$
    
      \If{\inter == \n}{
        \return $\p_{\ti}$
      }
    }
    
    \uElseIf{\p is \eT}{
    
      \inter $-= 1$
      
      \ti $\gets 2 * \p_{\unit} + \p_{\ti}$
      
      \li $\gets$ MakeTimePoints(\ti, $\p_{\unit}$, $\p_{\run}$, \li)
    }
    \ElseIf{\p is \finish}{
      \return \no
    }
  }
\end{algorithm}

\subsubsection{Proofs of the Geometric algorithm}
\algProof{
?
}{
It is clear that the algorithm will always terminate, since the algorithm terminates when it finds \comFin (which cannot be removed prematurely).
}{
The first loop of the algorithm takes $O(n\log(n))$ time, since it has to add $2n$ event points to the Priority Queue, where each insertion taking $O(\log(n))$.\\

After the first loop, there will at most be $2n + 1$ or $O(n)$ Event Points in the Priority Queue, since each time we encounter a \comEnd, two new Event Points are added to the Priority Queue.\\ 

It is clear that the second loop is dependent on how many Event Points are processed before \comFin, at time $1$ is encountered. For any given runner with speed $s$, it is clear that it will make a $s$ rounds before the ``fake'' runner arrives at the finishing line, and each time  a runner makes a complete tour of the track, the algorithm has to deal with 2 Event Points --- a \comStart\; and an \comEnd. The \comStart\; event takes $O(1)$ to process, while the \comEnd will take $O(\log(n))$ time, because of Lemma 1\todo{introduce Lemma 1 - it should say that there will always be O(n) points in the heap}. Therefore it is clear that if we let  $k$ be the sum of all the runners speeds. Since all the speeds are positive and pairwise different, it is easily seen that either $n \leq k$, which means $O(k\log(n))$.
}

\subsection{Number Theory}

A Number Theory alternative to the Geometric Algorithm can be made by applying the results of \cite{invis}. I have sketched the algorithm described from \cite{invis} below. \cite{invis} studies the the function 
$$
k(D) = \sup_{x \in \R}\min_{1 \leq i \leq k}\Vert x d_i \Vert
$$
where D is a set $\E{d_1, d_2, \ldots, d_k}$ of integers. They arrive at the result that given D, $k(D)$ ``is attained for $x_0 = a /(d_i + d_j)$ for some $i \neq j$ and some positive integer a''. (Theorem 6 in \cite{invis}). It is clear that $a < k$, since if $a = k$ then $a/k$ would evaluate to $1$ and then $\Vert \frac{a}{k} d\Vert, d \in D$ would evaluate to 0.

\todo{Make sure you have the correct algorithm} .

\begin{algorithm}[H]
  \caption{NumericalLonelyRunner}
  \highlights
  \SetKwData{and}{and}
  \Input{A list \s which contains n pairwise different speeds for n runners}
  \Output{A time \ti such that all n runners are at least $\frac{1}{n + 1}$ units away from the starting line, or a \no, indicating that no such time exists}
  
  $n \gets size(\s)$

  \For{$i \gets 1$ \KwTo $n-1$}{
    
    $d \gets \s_i$
 
    \For{$j \gets i + 1$ \KwTo $n$}{
      
      $d^{\prime} \gets \s_j$

      $k \gets d + d^{\prime}$
      
      \For{$a \leftarrow 1$ \KwTo $k-1$}{
        
        $testValid \gets true$
        
        \ForEach{$s \in \s$}{
          
          $testValid \gets testValid\; \and\; (\Vert \frac{a}{k} * s \Vert \geq \frac{1}{n+1} )$

          \If{!$testValid$} {
            break
            }
          
        }
        
        \If{testValid}{
          
          \return $\frac{a}{k}$
          
        }
      }
    }
  }
  \return \no
\end{algorithm}

\subsubsection{Proofs for the numerical algorithm}

\algProof{
In \cite{invis} the authors prove that if the Lonely Runner conjecture holds, then the algorithm will produce a solution to it. To take care of the possible case that the Lonely Runner conjecture does not hold, I have added the last return statement, ensuring that the correct answer will always be reported.
}{
It is clear that the algorithm will always terminate, $\Vert x \Vert $ is the only function call in the algorithm, and it will always terminate.
}{
It is clear that the meat of NumericalLonelyRunner is 4 nested for-loops. The 2 first loops are dependent on each other, the third is based on the speed for the two runners that have been chosen, and the last loop goes through all the runners.\\

The first 2 loops gives us $\sum_{i=1}^{n-1}\sum_{j=i+1}^{n}1 = n-1 + n-2 + \ldots 2 + 1$, which reduces to $O(n^2)$. The third loop must iterate $k-1$ times, where $k$ is the sum of 2 runner's speed.\\

If we are to give a worst case time, then it is clear $k$ must be the sum of the speed of the two fastest runners - so if we let $speed_1 = max(\comS)$ and $speed_2 = max(\comS \setminus \E{speed_1})$ then $k = speed_1 + speed_2$
And we go through the last loop $n$ times. Thus the entire algorithm runs in $O(k * n^3)$.
}

\subsection{Conclusion}

From the above we results we can see that neither of the algorithm are better in all cases. The Geometrical algorithm has a better run-time when we have many runners with low speeds, while the numerical algorithm is better when we have few runners with very high speeds. It is also clear that the numerical algorithm is far easier to implement, relying on no extra data structures than integers and decimal/floating numbers.

However, since they both are so short, and so easy to implement, I will try to implement them both and compare their run-times for a variety of configurations, in order to truly test their strengths and weaknesses. 

In this section I have presented two different \todo{decide whether to use numbers or words to define the size of sets} methods that can be used to decide whether equation \eqaref{eqa:lonelyRunner} is true for a given configuration. I have analysed their run-times and given proofs that they will always terminate, and that they will return the correct solution to equation \eqaref{eqa:lonelyRunner}, is one such exists, and otherwise inform the caller that no such value exists.
