\section{Possible methods}
\label{choiceOfMethod}

\subsection{Introduction}
I will in this subsection describe the different methods I have thought of using in this project, and discuss their merits and flaws.

\begin{description}
\item[Computational Geometry:] The intuitive description of the Lonely Runner conjecture (descibed in the Introduction) lends itself well to a geometrical interpretation, and leads me to wonder whether an algorithm based on this fact would be efficient.

\item[Number theoretic:] The Lonely Runner conjecture is, in its original formulation, a problem from number theory. It is therefore not reasonable to assume that a number theoretic approach could lead to great speedups in calculating a possible solution.
\end{description}

\subsection{Computational Geometry}
The following solution is based on the first intiutive description of the conjecture. Since we are interested when the runners are $\frac{1}{n + 1}$ units, or more, from the starting point, it is clear that we are interested when the runners are in the interval [$\frac{1}{n + 1}$, $\frac{n}{n + 1}$] on the track. However, we are not interested about the interval on the track, but rather, if a solution exist, the timepoint when all the runners are $\frac{1}{n+1}$. So one possible solution to the problem would be to calculate all the time intervals when a given runner is $\frac{1}{n+1}$ units from the start position. 

In that case it becomes a matter of finding out whether there is a overlap of all the intervals, for instance using a Plane Sweep Algorithm. Since the Lonely Runner Conjecture has not been proven, we have to make sure the algorithm will always terminate.  

\subsection{Number Theoretic}

