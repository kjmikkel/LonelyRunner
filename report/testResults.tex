\section{Test Results}
\label{test}
\todo{make tests and describe them}

\subsection{Introduction}
In this section I will introduce how I am going to test the algorithms, which parameters I am going to test them on, which software and hardware setup I have tested them on, description of the test themselves, and finally the results.

\subsection{Input parameters}
For the Lonely Runner conjecture there are 2 parameters we can check
on:
\begin{enumerate}
\item Number of runners
\item The speeds of the runners
\end{enumerate}

Since it is a requirement that all speeds are to be pairwise distinct,
it is not a directly obvious how to define the second
parameter. Given a sequence of speeds, how should we define which
speeds are to be used, and which are not? 

Should the number we give be
the maximum allowable speed (which means some tests have to be
excluded because the number of runners needed exceeds the maximum speed) or
should the number define the number of speeds (which means it becomes
harder to talk about how either algorithm handles certain classes of
numbers).

I have chosen to try both methods, since both plays a crucial role in
the time complexity of the respective algorithms, and in different ways.

\subsection{Test input}
In the following tests I will test 3 kinds of speeds for the runners
\begin{enumerate}
\item Prime numbers
\item Sequential numbers
\item Random numbers
\end{enumerate}

All of these will be positive, and all numbers in each set will be
pair-wise unique.




\subsection{Tests made on the Numerical method}
As I stated back in (section \ref{numtheory:algo}) the numerical
method's run-time is dependent on order that its input. I will
therefore make multiple tests:
\begin{enumerate}
\item Where the input will not be changed
\item Where the input is randomly permuted
\end{enumerate} 

\subsection{How to test the Algorithms}
Since a configuration of the Lonely Runner conjecture requires $n$ runners and each runner needs a distinct speed, it is clear that the parameters we must test the algorithms for are the number of runners, and their speed. Since the speeds have to be different, I will use the maximum speed to be the defining factor.

To do this test in practise I have chosen to do it in increments, with the number of runners being the deciding factor. If we have to test for $n$ runners, then we let the first configuration to test contain the first $n$ numbers, the second contain from ($n+1$)'th number to the ($2n+1$)'th number and so on.

For example if we have to test 10 runners for values up to 100 with sequential numbers, then the first array to be tested will be [2, 3, 4, $\ldots$, 9, 10, 11], the second array to be tested [12, 13, 14, $\ldots$, 19, 20, 21] and so on.

Since the Lonely Runner conjecture is about the relation between numbers, it would be interesting to test the algorithms in the cases where the numbers are pairwise prime. To do this I have simply calculated the prime numbers up to the maximum value, and used them as the speed.

As a stress test I have also tried calculating the sequential numbers for the 

\subsection{Hardware and software setup}
As mentioned in the implementation part I have implemented the software in C++ using g++ 4.4.3 compiler for Ubuntu. The test was run on a [] \todo{name computer type, hardware, os and so forth once final tests have been made}.

\subsection{Test made} 

\begin{description}
\item[Runners:] I will test both algorithms with 10, 50, 100, 500, 1000 and 2000
runners 
\item[Speeds:] For the sequential and the prime numbers, I will use
  500000 as the maximum speed. I will also randomly generate up to
  500000 pairwise distinct numbers and test them on both algorithms.
\end{description}
