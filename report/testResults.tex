\section{Tests}
\label{test}

\subsection{Introduction}
In this section I will introduce how I am going to test the algorithms, which parameters I am going to test them on, which software and hardware setup I have tested them on, description of the test themselves, and finally the results.

\subsection{Input parameters}
For the Lonely Runner conjecture there are 2 parameters we can check
on:
\begin{enumerate}
\item Number of runners
\item The speeds of the runners
\end{enumerate}

Since it is a requirement that all speeds are to be pairwise distinct,
it is not a directly obvious how to define the second
parameter. Given a sequence of speeds, how should we define which
speeds are to be used, and which are not? 

Should the parameter be the maximum allowable speed (which means some tests have to be
excluded because the number of runners is greater than the number of numbers below the the maximum speed) or should the parameter define the number of speeds used from a list (which means it becomes
harder to talk about how either algorithm handles certain classes of
numbers).

I have chosen to try both methods, since both plays a crucial role in
the time complexity of the respective algorithms, and in different ways.

\subsection{Test input}
In the following tests I will test 3 kinds of speeds for the runners
\begin{enumerate}
\item Prime numbers
\item Sequential numbers
\item Random numbers\footnote{Since the Lonely Runner conjecture is about the relation between numbers, it is interesting to test the algorithms in the cases where the numbers are pairwise prime. Hence the prime number test.}
\end{enumerate}

For the Prime, and Sequential number tests the second parameter will be the maximum speed, while for the random numbers I will use the second interpretation.

All values in all sets will be positive and within the sets all values are pair-wise unique.

\subsection{Tests made on the Numerical method}
As I stated in Section \ref{numtheory:algo}, the numerical
method's run-time is dependent on order that its input. I will
therefore make multiple tests:
\begin{enumerate}
\item Where the algorithm deals with the original input
\item Where the input is randomly permuted
\end{enumerate} 

\subsection{How to test the Algorithms}
Since a configuration of the Lonely Runner conjecture requires $n$ runners and each runner needs a distinct speed, it is clear that the parameters we must test the algorithms for are the number of runners, and their speed. When the tests deals with maximum speeds, I will use that as the label for the graph/table, and when the test deals with indexes, I will use the index as a label.

To do this test in practise I have chosen to do it with a sliding window, with the number of runners being the deciding factor. If we have to test for $n$ runners, then we let the first configuration to test contain the first $n$ numbers, the second contain from ($n+1$)'th number to the $2n$'th number and so on, until we cannot do this anymore.

For example if we have to test 10 runners for values up to 100 with sequential numbers, then the first array to be tested will be [1, 2, 3, $\ldots$, 8, 9, 10], the second array to be tested [11, 12, 13, $\ldots$, 18, 19, 20] and so on.



\subsection{Hardware and software setup}
As mentioned in the implementation part I have implemented the software in C++ using g++ 4.4.3 compiler for Debian. The test was run on a [] \todo{name computer type, hardware, os and so forth once final tests have been made}.

\subsection{Combinations of Parameters} 

\begin{description}
\item[Runners:] I will test both algorithms with 100, 500, 1000, 2000, 4000, 5000, 8000, 12000, 300000, 5000 runners 
\item[Speeds:] For the sequential and the prime numbers, I will use
  500000 as the maximum speed, and randomly generate 5000 pairwise distinct numbers and test them on both algorithms.
\end{description}

\subsection{Range test}
Using the Numerical algorithm I have tested all possible configurations for 20 runners with all speeds up to \maxNumbers.
