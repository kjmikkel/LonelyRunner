\section{Introduction}
\label{introduction}
\todo{Introduction}

The Lonely Runner conjecture states that given 

The most succinct description I have seen of the problem is found in \cite{ANote} which has the following description of the problem:\\

Given any n positive integers $w_1, w_2, \ldots, w_n$, there is a real number $x$ such that 
$$
\Vert w_i x\Vert \geq \frac{1}{n+1}
$$

for each $i = 1, 2, \ldots, n$, where for a real number $x$, $\Vert x \Vert$ is the distance from $x$ from the nearest integer.

There exist 2 equivalent descriptions, which are more colourful:
\begin{enumerate}
\item Assume there are $n$ runners at a circular track (with unit length), where every runner has pairwise, constant, different speed (which, without loss of generality can be assumed to be in $\N$ \cite{Bienia97flows.view-obstructions} \todo{Should I instead point to Zwei Sate uber inhomogene diophntische Approximation von Irrationalzahlen?}), there will then be a time where all the runners are at least $\frac{1}{n + 1}$ distance from the start line.\\

\item Instead of n runners, we now have n + 1 runners, then the conjecture states that there exists a point in time where all the runners are at least $\frac{1}{n + 1}$ units away from their nearest runner, everything else is the same \cite{Bienia97flows.view-obstructions}.\\
\end{enumerate}

\subsection{Scope and Limitations}
\label{scope}
\todo{Scope and limitations}
In this project I will not attempt to prove the Lonely Runner conjecture for all n. 
I will work to make the program give a result (within a reasonable time) with up to 1000 runners. While the program may work with more than 1000 runners, I will not dedicate any time to this.

\subsection{Expectations to the reader}
\label{expectations}
\todo{Expectations}
I expect the reader to be able to understand the Lonely Runner Conjecture in all its formulations, as well as being mathematically mature.

\subsection{Terminology}
\label{Termonolgy}
In this report I will refer to the track interval [$\frac{1}{n + 1}$: $\frac{n}{n+1}$] as being in the \zone.

\subsection{Learning targets and objectives}
\label{learning}
\todo{Learning targets - should this even be here?}

\subsection{Overview}
\todo{Make sure overview fits with the actual content}

In this section I give a quick overview of what the different sections in report will cover.
\begin{description}
\item[Section \ref{introduction}] The introduction to the report, which will introduce the subject and explain my goals
\item[Section \ref{choiceOfMethod}] A survey of the different methods for Author Attribution  
\item[Section \ref{method}] Explanation of the Authorship Attribution method used in my project
\item[Section \ref{implementation}] Description of the implementation details and an analysis of the scalability of the implementation
\item[Section \ref{test}] The test results
\item[Section \ref{results}] Interpretation of results
\item[Section \ref{conclusion}] The conclusion of the report  
\end{description}
