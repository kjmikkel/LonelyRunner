\section{Introduction}
\label{introduction}
\todo{Introduction}

There exist 2 equivalent interpretations of the Lonely Runner conjecture\footnote{The Lonely Runner conjecture got its name from \cite{Bienia97flows.view-obstructions7}} that are widely used\footnote{See \cite{ANote}}:
\begin{enumerate}
\item Imagine $n$ runners at a circular, unit length track\footnote{I.E. the track has a circumference of 1}, where every runner runs with a constant, pairwise different speeds\footnote{These speeds can, without loss of generality be assumed to be in $\N$ \cite{Bienia97flows.view-obstructions} - also see \ref{integerSpeeds}}, there will then be a time when all runners will be at least $\frac{1}{n + 1}$ units away from their common start-point.\\

\item Alternatively, imagine that we instead have n + 1 runners, then the conjecture states that there exists a point in time where all the runners are at least $\frac{1}{n + 1}$ units away from their nearest runner. Everything else is the same as in the first interpretation \cite{Bienia97flows.view-obstructions}.\\
\end{enumerate}

More formally the problem can be stated thus: 
Given any n positive integers $w_1, w_2, \ldots, w_n$, there is a real number $x$ such that 
\eqn{
\label{eqa:lonelyRunner} \Vert w_i x\Vert \geq \frac{1}{n+1}
}

for each $i = 1, 2, \ldots, n$, where for a real number $y$, $\Vert y \Vert$ is the distance from $x$ from the nearest integer. I owe this formulation of the problem to \cite{ANote}.

\subsection{Expectations to the reader}
\label{expectations}
\todo{Expectations}
I expect the reader to have read at lea be able to understand the Lonely Runner Conjecture in all its formulations, as well as being mathematically mature, and having a basic understanding of Plane Sweep algorithms.

\subsection{Scope and Limitations}
\label{scope}
\todo{Scope and limitations}
\begin{itemize}
\item In this project I will not attempt to prove the Lonely Runner conjecture for all n. 
\item I will work to make the program return a result within a reasonable time, for up to 1000 runners, and with speeds up to 1000. The final program should work with more than 1000 runners, and with speeds in excess to 1000, but I will not optimise it further. 
\item The program will include an interactive counterexample search system.
\item I will only focus on runners with integer speeds - see \ref{integerSpeeds} for an justification of this.
\end{itemize}

\subsection{Computer specifications}
\label{specs}\todo{Look at this once more}
Before I go any further I would like to define which kind of computer I would require for running the solutions I come up with. As mentioned above, the Lonely Runner Problem has direct applications for View-Obstruction problems and Colouring of regular chromatic graphs. None of people working with the previous has any special reason to work with any a exceedingly powerful computer, or computers with non-standard hardware.
 
Based on the above I would argue that any algorithm I devise and implement should be able to run on a standard computer with no special hardware (say hardware dedicated to fast floating point calculations). The minimum specs for the computer therefore becomes a 2 Ghz computer with more than 2 GB over the recommended amount of Ram for the OS the computer is running. 

I do however reserve the right to make it depend on the third-party software libraries - as long as these can be freely distributed and installed.

\subsection{Integer speeds}
In this section I will argue that I need only focus on integer speeds.
Let us assume there is a non-empty set S of runner speeds that are not integers. Since we are dealing with the speeds of runners, it is clear that it only makes sense either all elements in S belong to $\mathbb{Q}$ or where at least one of the elements in S is a irrational number:
\begin{description}
\item[Only rational numbers in S:] In this instance we can convert all the runner speeds into integers, by multiplying all speeds (also those not in S) with the product of the denominators, of all the rational numbers in S.
\item[Irrational numbers exist in S:]   
\end{description}

\subsection{Terminology}
\label{Termonolgy}
\begin{figure}[H]
  \centering
  \includegraphics[width=0.3\textwidth]{./images/circleZonePng.png}
  \caption{\label{circleZoneImg}An illustration of the Runner track. The greyed out part is the Zone}
\end{figure}

When forced to use a pronoun by the English language, I will use the male pronoun to the refer to the runners. This is done purely for convenience, and to make the text flow better. Not because I believe that male runners are more numerous or in any way inherently better than female runners.

\begin{defi}
\label{def:theZone}
In this report I will refer to the track interval [$\frac{1}{n + 1}$, $\frac{n}{n+1}$] as the Zone, and a runner who is in this interval, as ``being in the Zone''.
\end{defi}

\begin{defi}
\label{def:config}
Configurations of runners, will in this report refer to a set S of $n \in \N$ runners, where for all runners $r, r\prime \in S$, $r \neq r\prime$, $r_{speed}, r\prime_{speed} \in \N$, and $r_{speed} \neq r\prime_{speed}$.
\end{defi}

\begin{defi}
\label{def:fakeRunner}
When talking about the fake runner, I will be referring to a runner that is not actually in race, but rather a runner with speed 1, who I have introduced in order to have a convenient termination criterion for my algorithm. See section \ref{termination} page \pageref{termination} for an elaboration for the reasons behind it.
\end{defi}  

\subsection{Background material used}
\label{background}
For this project I have read the following papers dealing with the Lonely Runner conjecture: ``View-Obstruction Problems''\cite{Bienia97flows.view-obstructions}, ``The lonely runner problem with seven runners'' \cite{serra_thelonely}, ``Regular chromatic number and the lonely runner problem'' \cite{Barajas2007479}, ``View-Obstruction Problems'' \cite{springerlink:10.1007/BF01832623}, ``Tight Instances of the Lonely Runner'' \cite{Goddyn96tightinstances}, ``A Note on the Lonely Runner Conjecture'' \cite{ANote} and ``Invisible runners in finite fields'' \cite{invis}.

I have also used the follow support literature:
``Computational Geometry'' \cite{citeulike:3347056}, ``Uniform Distribution of Sequences'' \cite{uniform} and ``Kalkulus'' \cite{kalkulus}.

\subsection{Overview}
\todo{Make sure overview fits with the actual content}

In this section I give a quick overview of what the different sections in report will cover.
\begin{description}
\item[Section \ref{introduction}] The introduction to the report, which will introduce the subject and explain my goals
\item[Section \ref{choiceOfMethod}] A discussion of the two different approaches that can decide whether or not the Lonely Runner conjecture holds for a given configuration
\item[Section \ref{implementation}] Description of the implementation details and an analysis of the scalability of the implementation
\item[Section \ref{test}] The test results
\item[Section \ref{results}] Interpretation of results
\item[Section \ref{conclusion}] The conclusion of the report  
\end{description}
