\section{Introduction}
\label{introduction}
\todo{Introduction}

There exist 2 equivalent interpretations of the Lonely Runner conjecture that are widely used:\todo{have a footnote where it was named}
\begin{enumerate}
\item Assume there are $n$ runners at a circular, unit length, track, where every runner runs with a constant, pairwise different speeds\footnote{These speeds can, without loss of generality be assumed to be in $\N$ \cite{Bienia97flows.view-obstructions}}, there will then be a time when all runners will be at least $\frac{1}{n + 1}$ distance away from their common start-point.\\

\item Alternatively, imagine we have n + 1 runners, then the conjecture states that there exists a point in time where all the runners are at least $\frac{1}{n + 1}$ units away from their nearest runner. Everything else is the same as in the first interpretation \cite{Bienia97flows.view-obstructions}.\\
\end{enumerate}

More formally the problem can be stated thus:\\

Given any n positive integers $w_1, w_2, \ldots, w_n$, there is a real number $x$ such that 
\eqn{
\label{eqa:lonelyRunner} \Vert w_i x\Vert \geq \frac{1}{n+1}
}

for each $i = 1, 2, \ldots, n$, where for a real number $y$, $\Vert y \Vert$ is the distance from $x$ from the nearest integer. I owe this formulation of the problem to \cite{ANote}.

\subsection{Expectations to the reader}
\label{expectations}
\todo{Expectations}
I expect the reader to be able to understand the Lonely Runner Conjecture in all its formulations, as well as being mathematically mature, and having a basic understanding of Plane Sweep algorithms.

\subsection{Scope and Limitations}
\label{scope}
\todo{Scope and limitations}
In this project I will not attempt to prove the Lonely Runner conjecture for all n. 
I will work to make the program return a result within a reasonable time, for up to 1000 runners. While the final program should work with more than 1000 runners, I will not try to optimize the program further. The program will include an interactive counterexample search system.
One of the reasons I in this report will only focus on runners with \todo{spørg jakob om dette}.

\subsection{Terminology}
\label{Termonolgy}
In this report I will refer to the track interval [$\frac{1}{n + 1}$, $\frac{n}{n+1}$] as the Zone, and a runner who is in this interval, as being in the Zone.

\subsection{Background material used}
\label{background}
For this project I have read the following papers dealing with the Lonely Runner conjecture: ``View-Obstruction Problems''\cite{Bienia97flows.view-obstructions}, ``The lonely runner problem with seven runners'' \cite{serra_thelonely}, ``Regular chromatic number and the lonely runner problem'' \cite{Barajas2007479}, ``View-Obstruction Problems'' \cite{springerlink:10.1007/BF01832623}, ``Tight Instances of the Lonely Runner'' \cite{Goddyn96tightinstances}, ``A Note on the Lonely Runner Conjecture'' \cite{ANote} and ``Invisible runners in finite fields'' \cite{invis}.

I have also used the follow support literature:
``Computational Geometry'' \cite{citeulike:3347056} Algebra \todo{Fix cite to Algebra by Anders Thorup}, and ``Kalkulus'' \cite{kalkulus}.

\subsection{Overview}
\todo{Make sure overview fits with the actual content}

In this section I give a quick overview of what the different sections in report will cover.
\begin{description}
\item[Section \ref{introduction}] The introduction to the report, which will introduce the subject and explain my goals
\item[Section \ref{choiceOfMethod}] A discussion of the two different approaches that can decide whether or not the Lonely Runner conjecture holds for a given configuration
\item[Section \ref{implementation}] Description of the implementation details and an analysis of the scalability of the implementation
\item[Section \ref{test}] The test results
\item[Section \ref{results}] Interpretation of results
\item[Section \ref{conclusion}] The conclusion of the report  
\end{description}
