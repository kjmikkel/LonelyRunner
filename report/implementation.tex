\section{Implementation}
\label{implementation}

In this section I will try to describe the implementation choices I have made in this project.


\subsection{Choice of programming language}
I will in this section discuss the factors that decided my choice of programming languages for this project, as well as the final choice.

I have used the following parameters to decide my choice of programming language:
\begin{description}
\item[Speed:] The point of this project is to make a program that can not only verify whether or not a given configuration of the Lonely Runner conjecture holds, but also to strengthen our belief in the Lonely Runner conjecture. The program must therefore be able to test a large number of configurations in short time. I have based these on the results found at ``The Computer Language Benchmarks Game''\footnote{The site can be found at http://shootout.alioth.debian.org/u32q/which-programming-languages-are-fastest.php and was last accessed the 7/11 2010}.
\item[Memory footprint:] Memory is slightly less essential than speed, but if the memory footprint is smaller, then we can have more runners in a given configuration. A smaller memory footprint would also seem to indicate that the program will be faster, since less has to be allocated and cleaned up. 
\item[Already implemented data-structures:] I would rather avoid having to implement, debug, and optimise the Priority Queue, and instead relying on a built-in implementation that has proved itself to be free of bugs.
\item[My familiarity with the language:] Familiarity with the language would enable me to get create a bug free implementation of the algorithms faster.
\end{description}

Based on these criteria I will give each programming language a value from 1 to 3, with 3 being the highest, and 1 being the smallest. 

For this project I have considered 3 programming languages: Java, Haskell and C++.

\begin{tabular}{l|c|c|c|c}
        & Speed & Memory & Data-structures & Familiarity \\
\hline
Java    & 2     &  1     &  3              &   3  \\
Haskell & 1     &  2     &  1              &   1  \\
C++     & 3     &  3     &  3              &   2  \\
\end{tabular}

\subsubsection{Conclusion}
Based on this I would argue that the best choice would be C++ on the grounds that it clearly scores the highest, and that it will be easy for me to implement the Geometrical and Numerical algorithms.

\subsection{Problems during implementation}
In this section I will describe some of the major problems I encountered during my implementation of the algorithms, and what I have done to resolve them.

\subsubsection{Rounding errors}
Since the Lonely Runner conjecture in its nature deals with rational numbers, it is clear that this has to be taken into account when implementing it in a real programming language. During my earliest testing phases I quickly discovered that even if the Geometrical algorithm found a valid solution, it would not be valid according to \eqaref{eqa:lonelyRunner}. My solution to this was to make the check entirely discrete. As stated in \eqaref{eqa:lonelyRunner}, we have a Lonely Runner iff
\eqa{
\Vert w_i x\Vert \geq \frac{1}{n+1}
} 

The main problem here being $\Vert w_i x\Vert$ - however, since we can be sure that the solution will always be rational (since all the speeds are integers) we have that x can be written of the form $\frac{p}{q}$ for a given $p \in \N\cup\E{0}$ and $q \in \N$ (since all speeds must be positive and the runners cannot cover negative distance). So now we have

\begin{equation}
\label{compareRewriteOne}
\begin{split}
\Vert w_i * \frac{p}{q}\Vert &\geq \frac{1}{n+1}\\
\min(\frac{w_i * p \bmod q}{q}, 1 - \frac{w_i * p \bmod q}{q}) &\geq \frac{1}{n+1}\\
\min(\frac{w_i * p \bmod q}{q}, 1 - \frac{w_i * p \bmod q}{q}) * (n+1) &\geq 1
\end{split}
\end{equation}

Depending on whether $\frac{w_i * p \mod q}{q}$ or 1 - $\frac{w_i * p \mod q}{q}$ is the smallest value, the equation then becomes

\begin{equation}
\label{compareRewriteTwo}
\begin{split}
\frac{w_i * p \bmod q}{q} * (n+1) &\geq 1\\
(w_i * p \bmod q) * (n+1) &\geq q 
\end{split}
\end{equation}

or 

\begin{equation}
\label{compareRewriteThree}
\begin{split}
1 - \frac{w_i * p \bmod q}{q} * (n+1) &\geq 1\\
(q - (w_i * p \bmod q)) * (n+1) &\geq q
\end{split}
\end{equation}

\subsection{Use of real numbers in the algorithm}
As detailed in Section \ref{eventPoints} I choose to implement the comparison of the Event Points through integer comparison, on the grounds that it is much faster than working with floating numbers, on most hardware. It is clear from \eqaref{eqa:integerTime} that neither side of the expression can be wholly precomputed, as it requires the other Event Point's speed, meaning that this solution trades floating point comparison with several integer multiplications. It is however clear on inspection that avoiding floating point number greatly speeds up the computation.

\subsection{Scalability of implementation}
\label{scale}
An important question that has to be answered is how well the 2 algorithms scale. In this section I will interpret scaling in 2 senses:
\begin{enumerate}
\item How much time it requires to give an answer given a certain input size
\item How large input sizes the algorithm can handle
\end{enumerate}

For the first interpretation it is clear from Section \ref{proof_geo} and \ref{proof_num} that both algorithms have a bad worst case time complexity. These are however unlikely to be relevant during most verification attempts (as we have yet to find a counter example to the Lonely Runner conjecture after more than 2 decades), and therefore not really a good measurement. 

The next question to ask is whether or not the algorithms could be parallelised, to take advantage of the multiple cores we are seeing in today's modern computers. Due to its shared state (the number of runners in the Zone) and the iterative nature of the Geometrical algorithm, I do not believe that the main loop of the algorithm as implemented could be parallelised. However, as discussed in \ref{eventPoints}, there might be a variant where parts of the algorithm could be parallelised. The numerical algorithm on the other hand is ripe for paralleisation, all 4 loops can easily be parralised, having no real shared state (other than, since the algorithm is based on enlightened guesses.

Since the point of this program is to strengthen our trust in Lonely Runner conjecture, the maximum number of runners, and the maximum speed of each runner, that we are able to test is important. I will therefore in the following section analyse the bounds on these values, and discuss how I have improved these.

\subsubsection{Geometrical algorithm}
In the current implementation of the comparison operator is that there is a limit on the number of runners and their speeds. Let us observe  
\eqa{
\label{eqa:firstCompare}
(a + k * (n+1)) * s
}
 from \eqaref{eqa:integerTime}, where all the symbols has the same meaning as before. The possible large numbers are $a$, which in the worst case is $n$, $n$ itself and $s$. No matter which built in number type we choose to represent the above expression with, it is clear there exists values for $n$ and $s$ that will cause an arithmetic overflow.

\hide{
The best I have been able to come up with is rewriting the entire comparison to 
\eqa{
\label{eqa:secondCompare}
s_ia_j - s_ja_i \leq (n+1)(k_is_j - k_js_i)
}
which still requires $s$ to be multiplied with $a$, making the speed of the runner and the local position (which in the worst case is $n$) the bottleneck.
}

The question then becomes what the maximum allowable value becomes by \eqaref{eqa:firstCompare}. Since we are not sure whether or not the Lonely Runner conjecture holds, we must make sure that the expression does not cause arithmetic overflow, even if the algorithm for that configuration of runners will terminate by finding \comFin. We must therefore assume the maximum values for the variables that change - the local position $a$ and k, the number of times the runner has completed a lap. The biggest value of $a$ is $n$ and given a speed $s$, the runner can at most run $s$ rounds before the final runner arrives at the the finishing line. Let $d$ be the maximum positive number that a given number type can represent, then to find the maximum value of $s$ with respect to $n$ and $d$, we have the expression 
\eqa{
(n + s * (n+1)) * s &=& d\\
s^2(n+1) + sn &=& d\\
s^2(n+1) + sn -d &=& 0\\
}
It is clear that this a quadratic equation with $s$ as the unknown - using the standard technique this gets us
\eqa{
s = \frac{-n \pm \sqrt{n^2 + 4(n+1) * d}}{2(n+1)}
}
In practise we are only interested in
\eqa{
s = \lfloor\frac{-n + \sqrt{n^2 + 4(n+1) * d}}{2(n+1)}\rfloor
}
as the other answer would be negative, which has no meaning in this context.

So if $n = 1000$ and $d = 2^{32}-1$ (an unsigned 32 bit integer) then $s = 2070$.

If we try to find $n$ with regards to $d$ and $s$, then we have
\eqa{
s^2n + sn + s^2 &=& d\\
n(s^2 + s) + s^2 &=& d\\
n(s^2 + s) &=& d - s^2\\
n &=& \frac{d - s^2}{s^2 + s}
}
which becomes
\eqa{
n = \lfloor \frac{d - s^2}{s^2 + s} \rfloor
}

So if we let $s = 1000$ and $d = 2^{32}-1$ then $n = 4289$

This is rather constraining, but if we use a 64 bit integer was to be used instead, then combinations such as $n = 10^6$ and $s = 4294964$ would become possible.

Since I in Section \ref{specs} detailed that I wanted the program to run on a normal computer, which yet does not cover 64-bit architectures, I have made the comparisons, and other checks that might cause overflow, use arbitrary-precision arithmetic\footnote{Based on the NTL library, see Section \ref{third_party}} - limiting the possible comparisons only by the memory of the system, at the cost of speed.

\subsubsection{Numerical algorithm}
A problem in the numerical implementation is that no sum $k$ of any 2 speeds must exceed the maximum value of the numerical type used (say, $2^{32} - 1$ for 32 bit integers). Even if this is far better than the Geometrical algorithm, I have implemented arbitrary-precision arithmetic, in order to make the algorithms equal in this regard.

\subsection{Third party libraries used}
\label{third_party}
\begin{description}
\item[NTL arbitrary-precision arithmetic (v. 5.5.2 ):] A software library to allow for arbitrary precision sized integers in both algorithms. The library can be found at http://www.shoup.net/ntl/doc/tour.html (last visited 10/11/2010).
\item[The GNU Multiple Precision Arithmetic Library (GMP) (v. 5.0.1):] The NTL library has been compiled towards the GMP to increase its efficiency. The library can be found at http://gmplib.org/ (last visited 10/11/2010).  
\item[json-c (v. 0.9):] A C++ library used for importing and exporting data to/from the JSON file format. The library can be found at http://oss.metaparadigm.com/json-c/ (last visited 10/11/2010).
\end{description}
